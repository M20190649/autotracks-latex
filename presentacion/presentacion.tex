\documentclass[oneside]{beamer}


%*****************************************************************************%
%   Theme para la Presentación                                                %
%*****************************************************************************%
\usetheme[pageofpages=de,% String used between the current page and thetotal page count.
          bullet=circle,% Use circles instead of squares for bullets.
          titleline=true,% Show a line below the frame title.
          alternativetitlepage=true,% Use the fancy title page.
          titlepagelogo=../logo.png,% Logo for the first page.
          ]{Torino}
% Nouvelle is a green and red alternative to the chameleon color theme.
\usecolortheme{freewilly}
\providecommand*{\eqcolon}{\mathrel{=\!\raise.16ex\hbox{\footnotesize\!:}}} % para las definiciones

%*****************************************************************************%
%   Paquetes principales                                                      %
%*****************************************************************************%
\usepackage{amssymb, amsmath}
\usepackage{amsthm}
\usepackage{amsfonts}
\usepackage{float}
\usepackage{afterpage}
\usepackage{dsfont}
\usepackage{url}
\usepackage{color}
\usepackage{lettrine}
\usepackage{algorithm}
\usepackage{algpseudocode}
\usepackage{graphicx}
%\usepackage{subfig}
\usepackage{caption}
\usepackage{subcaption}
\usepackage{multimedia}
\usepackage{verbatim} % comentarios
\usepackage{pdfpages}
\usepackage[spanish]{babel}
\usepackage[utf8]{inputenc}
\usepackage{textcomp}

%Para determinar la fecha actual
\newcommand{\monthname}{\ifcase\month\or Enero\or Febrero\or
      Marzo\or Abril\or Mayo\or Junio\or Julio\or Agosto\or Septiembre\or
      Octubre\or Noviembre\or Diciembre\fi}

\newcommand{\thismonth}{\monthname,\ \the\year}


\author{\  \\ Autor del trabajo \\Autor 2}
\title{Aproximación colaborativa del tránsito vehicular basada en aplicaciones móviles}
\institute{Facultad Politécnica- UNA}
%pone la fecha de generación como portada
\date{\thismonth}


\begin{document}
%para forzar el total de número de páginas, se utiliza para no tener en cuenta las páginas del anexo
%\renewcommand{\inserttotalframenumber}{71}
\begin{frame}[t,plain]
\titlepage
\end{frame}

%TODO no sabemos como hacer
\begin{frame}[c]{Motivación y Definición del Problema}
	\begin{center}
	La congestión del tránsito es uno de los problemas más serios que enfrentan la mayoría de las zonas urbanas hoy en día. 
	
	\vspace{\baselineskip}
	\vspace{\baselineskip}
	Se debe encontrar una alternativa efectiva y de bajo costo para monitorear el estado del tránsito.
    \end{center}
\end{frame}

%TODO queremos sacar
\begin{frame}[c]{Propuesta}
    \begin{center}
    Se propone la implementación de un sistema que permita aproximar las condiciones del tránsito utilizando información proveída por dispositivos móviles en tiempo real.
    \end{center}
\end{frame}

\begin{frame}[c]{Objetivo General}
    \begin{center}
    Construir un sistema que permita recolectar
    y procesar información del tránsito vehicular a través de dispositivos móviles para aproximar el estado del tráfico en tiempo real.
    \end{center}
\end{frame}

\begin{frame}[c]{Objetivos Específicos}
    \begin{center}
        \begin{itemize}
        \item Analizar el estado del arte de las técnicas de recolección y análisis de información de tráfico.
        \item Diseñar la arquitectura para un sistema que utilice las técnicas analizadas.
        \item Implementar una aplicación móvil que permita recolectar datos de ubicación de los usuarios y distribuir la información de tránsito.
        \item Implementar un sistema centralizado capaz de recibir los datos de ubicación para determinar el estado del tránsito vehicular.
        \end{itemize}
    \end{center}
\end{frame}

\begin{frame}[t]{Resultados y discusión.}
    \begin{center}
        Resultados del trabajo.
    \end{center}
\end{frame}

\begin{frame}[t]{Conclusiones.}
    \begin{center}
        Conculsiones del trabajo.
    \end{center}
\end{frame}


\begin{frame}[t]{Conclusiones.\\\textit{Cumplimiento de Objetivos.}}
    \begin{center}
        \begin{itemize}
        \item \textit{Objetivo Especifico 1.}
        \pause
        \\Si.
        \pause
        \item \textit{Objetivo Especifico 2.}
        \pause
        \\Si.

        \item \textit{Objetivo Especifico 3.}
        \pause
        \\Si.
        \end{itemize}
    \end{center}
\end{frame}

\begin{frame}[t]{Trabajos futuros}
    \begin{center}
        \begin{itemize}
            \item Trabajo Futuro 1.
            \item Trabajo Futuro 2.
            \item Trabajo Futuro 3.
            \item Trabajo Futuro 4.
        \end{itemize}
    \end{center}
\end{frame}

\begin{frame}[c]{Aportes.}
    \begin{itemize}
        \item Aporte 1.
        \item Aporte 2.
        \item Aporte 3.
    \end{itemize}
\end{frame}

%~ \include{./propuesta/objetivos}
%~ \include{./propuesta/modelo-propuesto}
%~ \include{./resultados/resultados}
%~ \include{./conclusion/conclusion}
%~ \include{./conclusion/trabajos-futuros}
%~ \include{./conclusion/aportes}
%~ \include{./conclusion/despedida}
\begin{frame}[c]{Recolección de Datos de Tráfico}
	\begin{center}
		\huge{Recolección de Datos de Tráfico}
    \end{center}
\end{frame}

\begin{frame}[c]{Recolección de Datos de Tráfico}
	\large{Clasificación de Tecnologías de detección:}
	\vspace{\baselineskip}
	\begin{center}
	 \begin{itemize}	 	
	 	\item \textbf{Tecnologías \emph{in-situ}}: dispositivos instalados en caminos.
	 	\vspace{\baselineskip}
	 	\item \textbf{Tecnologías de \emph{Floating Car Data}}: dispositivos en vehículos.
	 \end{itemize}			
	\end{center}
\end{frame}

%TODO escalar figuras

\begin{frame}[c]{Recolección de Datos de Tráfico}
	\begin{figure}[h]
		\centering
		\input{figuras/in_situ.pdf_tex}
		\captionsetup{singlelinecheck=off}
		\caption[Configuraciones de detección intrusiva]{}
		\label{fig:intrusiva} 
	\end{figure}
\end{frame}

\begin{frame}[c]{Recolección de Datos de Tráfico}
	\begin{figure}[h]
		\centering
		\input{figuras/no_intrusivo.pdf_tex}
		\captionsetup{singlelinecheck=off}
		\caption[Configuraciones de detección no intrusiva]{}
		\label{fig:intrusiva} 
	\end{figure}
\end{frame}

\begin{frame}[c]{Recolección de Datos de Tráfico}
	\large{Tecnologías de \emph{Floating Car Data}:}
	\vspace{\baselineskip}
	\begin{center}
		\begin{itemize}	 	
			\item FCD basado en GPS	
			\vspace{\baselineskip}
			\item FCD basado en teléfonos móviles
		\end{itemize}			
	\end{center}
\end{frame}

\begin{frame}[c]{Recolección de Datos de Tráfico}
	\begin{figure}[h]
		\centering
		\input{figuras/fcd.pdf_tex}
		\captionsetup{singlelinecheck=off}
		\caption[Configuraciones FCD]{}
		\label{fig:intrusiva} 
	\end{figure}
\end{frame}


%\include{./anexos/anexos}

\end{document}
