\documentclass[oneside]{beamer}


%*****************************************************************************%
%   Theme para la Presentación                                                %
%*****************************************************************************%
\usetheme[pageofpages=de,% String used between the current page and thetotal page count.
          bullet=circle,% Use circles instead of squares for bullets.
          titleline=true,% Show a line below the frame title.
          alternativetitlepage=true,% Use the fancy title page.
          titlepagelogo=../logo.png,% Logo for the first page.
          ]{Torino}
% Nouvelle is a green and red alternative to the chameleon color theme.
\usecolortheme{freewilly}
\providecommand*{\eqcolon}{\mathrel{=\!\raise.16ex\hbox{\footnotesize\!:}}} % para las definiciones

%*****************************************************************************%
%   Paquetes principales                                                      %
%*****************************************************************************%
\usepackage{amssymb, amsmath}
\usepackage{amsthm}
\usepackage{amsfonts}
\usepackage{float}
\usepackage{afterpage}
\usepackage{dsfont}
\usepackage{url}
\usepackage{color}
\usepackage{lettrine}
\usepackage{algorithm}
\usepackage{algpseudocode}
\usepackage{graphicx}
%\usepackage{subfig}
\usepackage{caption}
\usepackage{subcaption}
\usepackage{multimedia}
\usepackage{verbatim} % comentarios
\usepackage{pdfpages}
\usepackage[spanish]{babel}
\usepackage[utf8]{inputenc}
\usepackage{textcomp}

\graphicspath{{figuras/}}

%Para determinar la fecha actual
\newcommand{\monthname}{\ifcase\month\or Enero\or Febrero\or
      Marzo\or Abril\or Mayo\or Junio\or Julio\or Agosto\or Septiembre\or
      Octubre\or Noviembre\or Diciembre\fi}

\newcommand{\thismonth}{\monthname,\ \the\year}


\author{\  \\ Autor del trabajo \\Autor 2}
\title{Aproximación colaborativa del tránsito vehicular basada en aplicaciones móviles}
\institute{Facultad Politécnica- UNA}
%pone la fecha de generación como portada
\date{\thismonth}


\begin{document}
%para forzar el total de número de páginas, se utiliza para no tener en cuenta las páginas del anexo
%\renewcommand{\inserttotalframenumber}{71}
\begin{frame}[t,plain]
\titlepage
\end{frame}

%TODO no sabemos como hacer
\begin{frame}[c]{Motivación y Definición del Problema}
	\begin{center}
	La congestión del tránsito es uno de los problemas más serios que enfrentan la mayoría de las zonas urbanas hoy en día. 
	
	\vspace{\baselineskip}
	\vspace{\baselineskip}
	Se debe encontrar una alternativa efectiva y de bajo costo para monitorear el estado del tránsito.
    \end{center}
\end{frame}

%TODO queremos sacar
\begin{frame}[c]{Propuesta}
    \begin{center}
    Se propone la implementación de un sistema que permita aproximar las condiciones del tránsito utilizando información proveída por dispositivos móviles en tiempo real.
    \end{center}
\end{frame}

\begin{frame}[c]{Objetivo General}
    \begin{center}
    Construir un sistema que permita recolectar
    y procesar información del tránsito vehicular a través de dispositivos móviles para aproximar el estado del tráfico en tiempo real.
    \end{center}
\end{frame}

\begin{frame}[c]{Objetivos Específicos}
    \begin{center}
        \begin{itemize}
        \item Analizar el estado del arte de las técnicas de recolección y análisis de información de tráfico.
        \item Diseñar la arquitectura para un sistema que utilice las técnicas analizadas.
        \item Implementar una aplicación móvil que permita recolectar datos de ubicación de los usuarios y distribuir la información de tránsito.
        \item Implementar un sistema centralizado capaz de recibir los datos de ubicación para determinar el estado del tránsito vehicular.
        \end{itemize}
    \end{center}
\end{frame}

%\begin{frame}[t]{Resultados y discusión.}
%    \begin{center}
%        Resultados del trabajo.
%    \end{center}
%\end{frame}
%
%\begin{frame}[t]{Conclusiones.}
%    \begin{center}
%        Conculsiones del trabajo.
%    \end{center}
%\end{frame}
%
%
%\begin{frame}[t]{Conclusiones.\\\textit{Cumplimiento de Objetivos.}}
%    \begin{center}
%        \begin{itemize}
%        \item \textit{Objetivo Especifico 1.}
%        \pause
%        \\Si.
%        \pause
%        \item \textit{Objetivo Especifico 2.}
%        \pause
%        \\Si.
%
%        \item \textit{Objetivo Especifico 3.}
%        \pause
%        \\Si.
%        \end{itemize}
%    \end{center}
%\end{frame}
%
%\begin{frame}[t]{Trabajos futuros}
%    \begin{center}
%        \begin{itemize}
%            \item Trabajo Futuro 1.
%            \item Trabajo Futuro 2.
%            \item Trabajo Futuro 3.
%            \item Trabajo Futuro 4.
%        \end{itemize}
%    \end{center}
%\end{frame}
%
%\begin{frame}[c]{Aportes.}
%    \begin{itemize}
%        \item Aporte 1.
%        \item Aporte 2.
%        \item Aporte 3.
%    \end{itemize}
%\end{frame}

%~ \include{./propuesta/objetivos}
%~ \include{./propuesta/modelo-propuesto}
%~ \include{./resultados/resultados}
%~ \include{./conclusion/conclusion}
%~ \include{./conclusion/trabajos-futuros}
%~ \include{./conclusion/aportes}
%~ \include{./conclusion/despedida}
\begin{frame}[c]
	\begin{center}
		\huge{Sistemas de Información Geográfica}
    \end{center}
\end{frame}

\begin{frame}[c]{Sistemas de Información Geográfica}
	\begin{center}
	 Se conocen como Sistemas de Información Geográfica (\emph{Geographic Information Systems} - GIS) a aquellos sistemas basados en computadoras diseñados para la recolección, mantenimiento, análisis y distribución de datos e información geográfica 	
	\end{center}
\end{frame}

\begin{frame}[c]{Sistemas de Información Geográfica}
	\large{Para almacenar información geográfica se requiere:}
	\vspace{\baselineskip}
	\begin{center}
		\begin{itemize}	 	
			\item Seleccionar un sistema de coordenadas.
			\item Seleccionar un modelo de datos.
			\item Utilizar una base de datos espacial.
		\end{itemize}			
	\end{center}
\end{frame}

\begin{frame}[c]{Sistemas de Coordenadas}
	Para	definir el sistema de coordenadas se tienen en cuenta:
	\vspace{\baselineskip}
	\begin{itemize}	 	
		\item El tamaño y la forma de la Tierra (modelada como un elipsoide)	
		\item Elementos de
		referencia a partir de los cuales se puede determinar la ubicación de cualquier otro elemento
	\end{itemize}			
	\vspace{\baselineskip}	
	Ejemplos:
	\vspace{\baselineskip}
	\begin{itemize}	 	
		\item \emph{North American
		Datum} de 1927 (NAD27)
		\item \emph{North American Datum} de 1983
		(NAD83)
		\item \emph{World Geodetic System} de 1984 (WGS84)
	\end{itemize}			
\end{frame}

\begin{frame}[c]{Modelos de Datos}
	Definen las estructuras de datos abstractas que son almacenadas en las bases de datos espaciales.
	\vspace{\baselineskip}
	\begin{itemize}	
		\item \textbf{Modelo de Datos Vectorial:} Representación de puntos, líneas y polígonos.	
		\item \textbf{Modelo de Datos Raster:} Espacio dividido en colección de celdas
		rectangulares.
		\item \textbf{Modelos de Datos de Red:} Modelo vectorial que incluye información topológica.
		\item \textbf{Red Irregular de Triángulos (TIN):} Representa superficies 3D como triángulos adyacentes.
	\end{itemize}			
\end{frame}

%TODO Agregar frame con figuras de modelos de datos

\begin{frame}[c]{Bases de Datos Espaciales}
	Son aquellas que definen tipos de datos y funciones para representar y guardar objetos geométricos en tablas regulares de bases de datos.

	\vspace{\baselineskip}
	Ejemplos:

	\vspace{\baselineskip}
	\begin{itemize}	 	
		\item PostgreSQL/PostGis.	
		\item SQLite/SpatiaLite.
		\item Oracle Spatial.
	\end{itemize}			
\end{frame}


\begin{frame}[c]{Tráfico}
	\begin{center}
		\huge{Tráfico}
    \end{center}
\end{frame}

\begin{frame}[c]{Tráfico}
	\begin{center}
		\large{El tráfico vehicular, o simplemente tráfico, es el fenómeno causado por el flujo de vehículos en una vía, calle o autopista.}
	\end{center}
\end{frame}

\begin{frame}[c]{Tráfico}
	\large{Medidas del Flujo de Tráfico}
	\vspace{\baselineskip}
	\begin{center}
		\begin{itemize}	 	
			\item Velocidad
			\vspace{\baselineskip}
			\item Volumen
			\vspace{\baselineskip}
			\item Densidad
		\end{itemize}			
	\end{center}
\end{frame}

\begin{frame}[c]{Recolección de Datos de Tráfico}
	\begin{center}
		\huge{Recolección de Datos de Tráfico}
    \end{center}
\end{frame}

\begin{frame}[c]{Recolección de Datos de Tráfico}
	\large{Clasificación de Tecnologías de detección:}
	\vspace{\baselineskip}
	\begin{center}
	 \begin{itemize}	 	
	 	\item \textbf{Tecnologías \emph{in-situ}}: dispositivos instalados en caminos.
	 	\vspace{\baselineskip}
	 	\item \textbf{Tecnologías de \emph{Floating Car Data}}: dispositivos en vehículos.
	 \end{itemize}			
	\end{center}
\end{frame}

%TODO escalar figuras

\begin{frame}[c]{Recolección de Datos de Tráfico}
	\begin{figure}[h]
		\centering
		\input{figuras/in_situ.pdf_tex}
		\captionsetup{singlelinecheck=off}
		\caption[Configuraciones de detección intrusiva]{}
		\label{fig:intrusiva} 
	\end{figure}
\end{frame}

\begin{frame}[c]{Recolección de Datos de Tráfico}
	\begin{figure}[h]
		\centering
		\input{figuras/no_intrusivo.pdf_tex}
		\captionsetup{singlelinecheck=off}
		\caption[Configuraciones de detección no intrusiva]{}
		\label{fig:intrusiva} 
	\end{figure}
\end{frame}

\begin{frame}[c]{Recolección de Datos de Tráfico}
	\large{Tecnologías de \emph{Floating Car Data}:}
	\vspace{\baselineskip}
	\begin{center}
		\begin{itemize}	 	
			\item FCD basado en GPS	
			\vspace{\baselineskip}
			\item FCD basado en teléfonos móviles
		\end{itemize}			
	\end{center}
\end{frame}

\begin{frame}[c]{Recolección de Datos de Tráfico}
	\begin{figure}[h]
		\centering
		\input{figuras/fcd.pdf_tex}
		\captionsetup{singlelinecheck=off}
		\caption[Configuraciones FCD]{}
		\label{fig:intrusiva} 
	\end{figure}
\end{frame}
\begin{frame}[c]
	\begin{center}
		\huge{Algoritmos de \emph{Map Matching} (MM)}
    \end{center}
\end{frame}

\begin{frame}[c]{Definición del problema de MM}
	\large
	\begin{itemize}	 	
		\item $p$: Muestra o punto (localización, velocidad, dirección, etc.)
		\item $T$: Conjunto ordenado de puntos (trayectoria)
		\item $V$: Conjunto de vértices (intersecciones de calles)
		\item $E$: Conjunto de aristas (segmentos de calles)
		\item $G(V,E)$: Grafo o red de calles
		\item $R$: Secuencia ordenada de aristas (camino reconstruido)
	\end{itemize}			
\end{frame}

\begin{frame}[c]{Definición del problema de MM}
	\Large
	\begin{center}
		\emph{Dada una trayectoria $T$ y una red de calles $G(V,E)$, encontrar el camino $R$ que hace coincidir a $T$ con su reconstrucción más realista sobre $G(V,E)$}		
	\end{center}
\end{frame}

\begin{frame}[c]{Definición del problema de MM}
	\Large
	\begin{center}
		TODO Mostrar figura
	\end{center}
\end{frame}

\begin{frame}[c]{Clasificación de algoritmos de MM}
	\Large
	\begin{itemize}	 	
		\item De acuerdo a la técnica de implementación:
		\begin{itemize}
			\large
			\item \emph{geométricos}
			\item \emph{topológicos}
			\item \emph{estadísticos}
			\item \emph{avanzados}
		\end{itemize}
	\end{itemize}
\end{frame}

\begin{frame}[c]{Clasificación de algoritmos de MM}
	\Large
	\begin{itemize}	 	
		\item De acuerdo al momento del procesamiento:
		\begin{itemize}
			\large
			\item \emph{incrementales} u \emph{on-line}
			\item \emph{globales} u \emph{off-line}
		\end{itemize}
		\vspace{\baselineskip}
		\item De acuerdo a la frecuencia de muestreo:
		\begin{itemize}
			\large
			\item \emph{alta frecuencia}
			\item \emph{baja frecuencia}
		\end{itemize}
	\end{itemize}
\end{frame}


%\include{./anexos/anexos}

\end{document}
