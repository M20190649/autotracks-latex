\begin{frame}[c]
	\begin{center}
		\huge{Sistemas de Información Geográfica}
    \end{center}
\end{frame}

\begin{frame}[c]{Sistemas de Información Geográfica}
	\begin{center}
	 Se conocen como Sistemas de Información Geográfica (\emph{Geographic Information Systems} - GIS) a aquellos sistemas basados en computadoras diseñados para la recolección, mantenimiento, análisis y distribución de datos e información geográfica 	
	\end{center}
\end{frame}

\begin{frame}[c]{Sistemas de Información Geográfica}
	\large{Para almacenar información geográfica se requiere:}
	\vspace{\baselineskip}
	\begin{center}
		\begin{itemize}	 	
			\item Seleccionar un sistema de coordenadas.
			\item Seleccionar un modelo de datos.
			\item Utilizar una base de datos espacial.
		\end{itemize}			
	\end{center}
\end{frame}

\begin{frame}[c]{Sistemas de Coordenadas}
	Para	definir el sistema de coordenadas se tienen en cuenta:
	\vspace{\baselineskip}
	\begin{itemize}	 	
		\item El tamaño y la forma de la Tierra (modelada como un elipsoide)	
		\item Elementos de
		referencia a partir de los cuales se puede determinar la ubicación de cualquier otro elemento
	\end{itemize}			
	\vspace{\baselineskip}	
	Sistemas de coordenadas existentes:
	\vspace{\baselineskip}
	\begin{itemize}	 	
		\item \emph{North American
		Datum} de 1927 (NAD27)
		\item \emph{North American Datum} de 1983
		(NAD83)
		\item \emph{World Geodetic System} de 1984 (WGS84)
	\end{itemize}			
\end{frame}

\begin{frame}[c]{Modelos de Datos}
	Definen las estructuras de datos abstractas que son almacenadas en las bases de datos espaciales.
	\vspace{\baselineskip}
	\begin{itemize}	
		\item \textbf{Modelo de Datos Vectorial:} Representación de puntos, líneas y polígonos.	
		\item \textbf{Modelo de Datos Raster:} Espacio dividido en colección de celdas
		rectangulares.
		\item \textbf{Modelos de Datos de Red:} Modelo vectorial que incluye información topológica.
		\item \textbf{Red Irregular de Triángulos (TIN):} Representa superficies 3D como triángulos adyacentes.
	\end{itemize}			
\end{frame}

%TODO Agregar frame con figuras de modelos de datos

\begin{frame}[c]{Bases de Datos Espaciales}
	Son aquellas que definen tipos de datos y funciones para representar y guardar objetos geométricos en tablas regulares de bases de datos.

	\vspace{\baselineskip}
	Ejemplos:

	\vspace{\baselineskip}
	\begin{itemize}	 	
		\item PostgreSQL/PostGis.	
		\item SQLite/SpatiaLite.
		\item Oracle Spatial.
	\end{itemize}			
\end{frame}

