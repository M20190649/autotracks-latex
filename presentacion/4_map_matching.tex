\begin{frame}[c]
	\begin{center}
		\huge{Algoritmos de \emph{Map Matching} (MM)}
    \end{center}
\end{frame}

\begin{frame}[c]{Definición del problema de MM}
	\large
	\begin{itemize}	 	
		\item $p$: Muestra o punto (localización, velocidad, dirección, etc.)
		\item $T$: Conjunto ordenado de puntos (trayectoria)
		\item $V$: Conjunto de vértices (intersecciones de calles)
		\item $E$: Conjunto de aristas (segmentos de calles)
		\item $G(V,E)$: Grafo o red de calles
		\item $R$: Secuencia ordenada de aristas (camino reconstruido)
	\end{itemize}			
\end{frame}

\begin{frame}[c]{Definición del problema de MM}
	\Large
	\begin{center}
		\emph{Dada una trayectoria $T$ y una red de calles $G(V,E)$, encontrar el camino $R$ que hace coincidir a $T$ con su reconstrucción más realista sobre $G(V,E)$}		
	\end{center}
\end{frame}

\begin{frame}[c]{Definición del problema de MM}
	\Large
	\begin{center}
		TODO Mostrar figura
	\end{center}
\end{frame}

\begin{frame}[c]{Clasificación de algoritmos de MM}
	\Large
	\begin{itemize}	 	
		\item De acuerdo a la técnica de implementación:
		\begin{itemize}
			\large
			\item \emph{geométricos}
			\item \emph{topológicos}
			\item \emph{estadísticos}
			\item \emph{avanzados}
		\end{itemize}
	\end{itemize}
\end{frame}

\begin{frame}[c]{Clasificación de algoritmos de MM}
	\Large
	\begin{itemize}	 	
		\item De acuerdo al momento del procesamiento:
		\begin{itemize}
			\large
			\item \emph{incrementales} u \emph{on-line}
			\item \emph{globales} u \emph{off-line}
		\end{itemize}
		\vspace{\baselineskip}
		\item De acuerdo a la frecuencia de muestreo:
		\begin{itemize}
			\large
			\item \emph{alta frecuencia}
			\item \emph{baja frecuencia}
		\end{itemize}
	\end{itemize}
\end{frame}