\chapter{Map Matching}

El proceso de Map Matching (MM) es utilizado en una gran variedad de servicios y aplicaciones SIG, como la predicción de trayectorias de los usuarios \cite{eisner2011algorithms}, los sistemas de navegación en vehículos \cite{kim2001adaptive}, el control del estado del tránsito en tiempo real \cite{thiagarajan2010cooperative, thiagarajan2009vtrack} y muchos otros. En este capítulo se hace un extenso análisis de los algoritmos existentes hoy en día.

\section{Definición del problema}

Antes de realizar una definición formal del problema primeramente debemos definir algunos conceptos básicos.

\textbf{Punto}: Un punto es el conjunto de todos las mediciones recolectados por el vehículo de prueba en un instante específico del tiempo. Estas mediciones deben incluir como mínimo la localización del vehículo y el tiempo en que fue tomada dicha localización. También puede incluir la dirección de desplazamiento, la velocidad y otros posibles datos recolectados mediante los sensores del vehículo.

\textbf{Trayectoria}: Una trayectoria es una secuencia ordenada de puntos pertenecientes al mismo vehículo. La trayectoria corresponde únicamente a los puntos recolectados durante un viaje del vehículo.

\textbf{Red de calles}: Una red de calles es un grafo dirigido que representa la forma y las propiedades del sistema de calles de un área geográfica en particular. Los vértices del grafo describen las intersecciones entre las calles y las aristas representan la forma y los atributos de las calles.

\textbf{Camino reconstruido}: Un camino reconstruido es una secuencia ordenada de calles conectadas entre sí a través de las cuáles el vehículo pudo haber transitado.

El problema de MM puede definirse de la siguiente forma: \emph{Dada una trayectoria T y una red de calles R, encontrar el camino C que hace coincidir a T con su reconstrucción más realista sobre R.}

En la \Cref{fig:map-matching} se puede apreciar en rojo la trayectoria de un vehículo sobre una red de calles y en verde el posible camino reconstruido por el cual transitó el vehículo.

\begin{figure}[h]
	\centering
	\input{capitulos/4/figuras/figura41.pdf_tex}
	\caption[Trayectoria y camino reconstruido mediante MM]{Ejemplo de trayectoria (rojo) y camino reconstruido (verde) mediante MM}
	\label{fig:map-matching} 
\end{figure}

\section{Clasificación de algoritmos de MM}

Los algoritmos de MM pueden clasificarse de acuerdo a la técnica utilizada en su implementación en: \emph{geométricos}, \emph{topológicos}, \emph{estadísticos} y \emph{avanzados}; de acuerdo al momento en que se realiza el procesamiento en: \emph{incrementales} y \emph{globales}; y de acuerdo a la frecuencia con que se toman los puntos en: de \emph{baja frecuencia} y de \emph{alta frecuencia}.

De acuerdo a la información y las técnicas utilizadas en su implementación los algoritmos de MM se pueden clasificar en: \begin{enumerate*}[1)]\item \emph{geométricos}, que utilizan solo la información de posición y distancia entre las calles y puntos \cite{white2000some}, son sencillos de implementar y suelen utilizarse como paso inicial en la implementación de otros algoritmos; \item \emph{topológicos}, que incorporan información topológica de la red de calles \cite{lou2009map,yuan2010interactive,greenfeld2002matching,quddus2003general}, utilizan información sobre la conectividad, restricciones de giro, sentido y límite de velocidad de las calles; \item \emph{estadísticos}, que definen regiones de probabilidad alrededor de cada punto y analizan los tramos dentro de dichas regiones \cite{ochieng2009map}, estos análisis estadísticos también se utilizan como parte de otros algoritmos y \item \emph{avanzados}, que combinan diversas técnicas geométricas, topológicas y estadísticas con otros conceptos avanzados como Filtros de Kalman, Modelos Ocultos de Markov, Lógica Difusa, entre otros \cite{thiagarajan2009vtrack,quddus2006high,thiagarajan2011accurate,fang2011enacq}\end{enumerate*}.

De acuerdo al momento en el que se realiza el procesamiento de los datos, existen dos categorías: \begin{enumerate*}[1)] \item los algoritmos \emph{locales} u \emph{on-line}, que realizan el map-matching cada vez que se obtiene un nuevo punto \cite{thiagarajan2009vtrack,thiagarajan2011accurate,greenfeld2002matching,quddus2003general,quddus2006high}, se utilizan en aplicaciones de tiempo real como asistentes personales de navegación y \item los algoritmos \emph{globales} u \emph{off-line}, que realizan el map-matching luego de que se han recolectado todos los puntos \cite{lou2009map,yuan2010interactive}, se utilizan en aplicaciones de análisis de tráfico o estudios sobre el comportamiento de usuarios\end{enumerate*}. Existen algoritmos que pueden ser adaptados para funcionar de manera local o global dependiendo de la aplicación para la que se utilizan.

Dependiendo de la frecuencia de muestreo se pueden identificar dos categorías: \begin{enumerate*}[1)] \item los algoritmos para \emph{alta frecuencia} o \emph{high-sampling}, que típicamente trabajan con intervalos de muestra en el rango de los pocos segundos y generalmente se ejecutan de forma on-line \cite{greenfeld2002matching,quddus2003general,quddus2006high} y \item los algoritmos para \emph{baja frecuencia} o \emph{low-sampling},  que funcionan para intervalos de muestreo de varios minutos y se ejecutan generalmente de forma off-line \cite{lou2009map,yuan2010interactive}. \end{enumerate*} En general todos los algoritmos pueden ser alimentados con muestras de alta o baja frecuencia pero ciertos algoritmos dejan de ser efectivos a medida que disminuye la frecuencia de muestreo.

\subsection{Algoritmos geométricos}

Los algoritmos geométricos fueron los primeros en ser desarrollados, son los más simples y rápidos pero a la vez son los más propensos a errores. Estos algoritmos tienen en cuenta únicamente la posición de los puntos y de los vértices y aristas que conforman la red de calles \cite{quddus2007current}.

El algoritmo más sencillo se denomina de \emph{punto a punto} \cite{white2000some}, consiste en buscar para cada punto, el vértice dentro del mapa que sea el más cercano a dicho punto. Esta técnica es muy propensa a errores puesto que no se tiene en cuenta la conectividad entre los vértices seleccionados. Esta técnica implícitamente favorece a aquellas calles que tengan una mayor densidad de vértices en la red de calles, lo que puede llevar a resultados incorrectos. En la \Cref{fig:punto-a-punto} se puede apreciar cómo el punto $p_0$ es incorrectamente asociado al vértice $b_1$ cuando realmente está más próximo a la línea comprendida entre los vértices $a_0$ y $a_1$.

\begin{figure}[h]
	\centering
	\input{capitulos/4/figuras/figura42.pdf_tex}
	\caption{Error en el algoritmo de punto-a-punto}
	\label{fig:punto-a-punto} 
\end{figure}

Otra técnica utilizada se denomina \emph{punto a curva} \cite{white2000some}, y consiste en buscar para cada punto, la calle (curva) del mapa más cercano a dicho punto. Como tampoco se tiene en cuenta la conectividad entre las calles, se pueden tener los mismos errores que con la técnica anterior. Esta técnica no es efectiva en zonas urbanas que tienen una mayor densidad de calles debido al margen de error de las localizaciones. En la \Cref{fig:punto-a-curva} se puede apreciar cómo el punto $p_4$ es incorrectamente asignado a la línea comprendida entre los vértices $b_0$ y $a_0$ debido a que la distancia a esta línea es más corta que la distancia a la línea comprendida entre $a_0$ y $a_1$.

\begin{figure}[h]
	\centering
	\input{capitulos/4/figuras/figura43.pdf_tex}
	\caption{Error en el algoritmo de punto-a-curva}
	\label{fig:punto-a-curva} 
\end{figure}

La última técnica dentro de esta categoría es conocida como \emph{curva a curva} \cite{white2000some}, y consiste en  utilizar la técnica de punto a punto para identificar un vértice candidato para un punto, luego se seleccionan todos los tramos del mapa que se originan en dicho punto y se calcula la distancia entre cada tramo y el tramo comprendido entre el punto actual y un punto siguiente, utilizando alguna medida de distancia. El tramo del mapa que resulta ser el más cercano al tramo comprendido entre los puntos es seleccionado como el tramo real que recorre el vehículo. Un tramo puede estar compuesta por una o más secciones de una calle en el mapa. La forma específica en que se definen estos tramos y la función que calcula la distancia puede variar entre implementaciones.

\subsection{Algoritmos topológicos}

En un SIG se conoce como \emph{topología} a la relación entre las distintas formas geométricas (puntos, líneas, polígonos), entre estos elementos pueden definirse relaciones de adyacencia y conectividad. Los mapas de calles se representan generalmente como puntos y líneas, las líneas representan secciones de calles y los puntos representan intersecciones entre las calles. Además, los mapas digitales cuentan con información adicional sobre las calles como ser los límites de velocidad, restricciones de giro y los sentidos de las calles. Todos los algoritmos que incorporan este tipo de información en la construcción del trayecto del vehículo se conocen como algoritmos topológicos \cite{quddus2007current}.

El algoritmo topológico desarrollado en \cite{greenfeld2002matching} utiliza distintos criterios de similaridad para determinar cual es la mejor calle candidata para cada punto. Los criterios utilizados son: \begin{enumerate*}[a)] \item la similaridad en la orientación (el grado de paralelismo entre dos puntos consecutivos y la calle candidata) \item la proximidad entre la localización del punto y la calle candidata y \item el tamaño del ángulo comprendido entre la dirección de desplazamiento reportada por la localización (bearing) y la dirección de la calle candidata \end{enumerate*}. Para cada calle candidata se realiza una suma ponderada de los criterios de similaridad y se elige como mejor candidata a aquella con la mayor suma. El mismo procedimiento es utilizado en \cite{quddus2003general} pero se agrega información adicional como la velocidad del vehículo y la posición relativa del punto con respecto al vértice más cercano. Ambos métodos requieren que se identifique correctamente la primera calle, para luego ir eligiendo las calles que tienen conexión con la misma. Un fallo en la elección inicial puede llevar a resultados incorrectos.

%TODO agreger figura con pseudocódigo?

El algoritmo desarrollado en \cite{lou2009map}, conocido como ST-Matching, utiliza información geográfica y topológica para asignar un valor numérico a cada camino posible y luego selecciona al camino con el mayor valor. Para cada punto de la trayectoria se obtiene una lista de posibles vértices candidatos utilizando la técnica de punto a curva, cada vértice candidato correspondiente a un punto está conectado a todos los vértices candidatos correspondientes al siguiente punto, obteniendo así el conjunto de todos los caminos posibles. Para asignar el valor numérico se tienen en cuenta dos tipos de análisis, \begin{enumerate*}[a)] \item el \emph{análisis espacial} y \item el \emph{análisis temporal}\end{enumerate*}, que dan nombre al algoritmo. En el \emph{análisis espacial} se calcula una \emph{probabilidad de observación} para cada vértice candidato y una \emph{probabilidad de transmisión} entre cada par de vértices candidatos consecutivos. En el análisis temporal se calcula la similaridad entre la velocidad promedio entre dos vértices candidatos y las restricciones de velocidad del camino comprendido entre los puntos. Otros trabajos posteriores han agregado diversas mejoras al algoritmo original, como un proceso de detección de localizaciones inválidas \cite{sakic2012map}, la normalización del cálculo de la probabilidad de transmisión y la prevención de bucles en el camino final obtenido \cite{budigm2012algorithm}.

%TODO agreger figura con gráfico del grafo dirigido acíclico?

En \cite{yuan2010interactive} se propone un algoritmo basado en ST-Matching que incorpora ademas el concepto de “voto interactivo” para modelar la influencia mutua que tienen entre sí todos los puntos de la trayectoria del vehículo (a mayor distancia entre candidatos, menor la influencia). Para cada punto candidato existe un conjunto de caminos posibles que pasan por él, el objetivo es determinar cuál de los caminos es el óptimo para cada punto candidato, luego cada punto candidato vota por su “mejor camino” y finalmente se selecciona el camino óptimo global de acuerdo al resultado de esta votación.

Ambos algoritmos, \cite{lou2009map} y \cite{yuan2010interactive}, están específicamente diseñados para trabajar de forma off-line y con una baja frecuencia de muestreo. En \cite{lou2009map} se  menciona que es posible utilizar el algoritmo en aplicaciones on-line definiendo una “ventana” de localizaciones para las cuales se realiza el procedimiento de map-matching. En \cite{sakic2012map} se realizaron varias adaptaciones para utilizar el algoritmo en una aplicación de tiempo real, definiendo un tamaño de ventana de una localización, efectivamente convirtiendo el algoritmo global en un algoritmo incremental.

\subsection{Algoritmos estadísticos}

Los algoritmos estadísticos, también conocidos como probabilísticos, son aquellos que definen regiones de “confiabilidad” alrededor de las localizaciones y seleccionan una calle candidata de entre todas aquellas que están dentro de esta región \cite{quddus2007current}. Para determinar la región de confianza se tienen en cuenta los posibles errores originados por los sensores utilizados para obtener la localización y los errores en el mapa digital. Estos algoritmos también pueden utilizar información geométrica y topológica de la red de calles.

El algoritmo desarrollado en \cite{ochieng2009map} toma en cuenta diversas fuentes de error asociadas con los sensores de localización, la trayectoria anterior del vehículo, la información topológica de las calles (conectividad y orientación de las calles), e información sobre la velocidad y la dirección del vehículo. Este algoritmo está específicamente diseñado para aplicaciones con requerimientos de tiempo real y con una alta frecuencia de muestreo. El algoritmo se divide en dos partes, \begin{enumerate*}[a)]
\item el proceso de inicial de selección (Initial Matching Process), y \item el proceso subsecuente de selección (Subsequent Matching Process)\end{enumerate*}. 

En el \emph{proceso inicial} se define alrededor de la primera localización una región de confiabilidad rectangular o elíptica, si dentro de esta región no existe ningún candidato se asume que el vehículo está fuera de la red de calles, si existe más de un candidato se utiliza información sobre la conectividad de los candidatos y la dirección de desplazamiento del vehículo para determinar el candidato más apropiado. El \emph{proceso subsecuente} es utilizado para determinar si la siguiente localización el vehículo sigue viajando o no por la misma calle, para ello se intenta identificar si el vehículo ha hecho alguna maniobra de giro o si está atravesando una intersección de calles y en caso de que se detecte alguna de estas dos condiciones se vuelve a realizar el proceso inicial de selección para determinar la siguiente calle sobre la que esta viajando el vehículo.

%TODO incluir gráfico con pseudocódigo

\subsection{Algoritmos avanzados}

Los algoritmos avanzados son aquellos que combinan diversas técnicas topológicas, geometricas y estadisticas con otras técnicas más avanzadas. Las técnicas más utilizadas en los algoritmos implementados en los últimos años son la Lógica Difusa y los Modelos Ocultos de Markov. Explicaremos brevemente estos conceptos antes de continuar con los algoritmos que los implementan.

Se conocen como Procesos de Markov a aquellos procesos estocásticos (procesos no-determinísticos) que cumplen con la condición de que la probabilidad de transición entre dos estados (la probabilidad de pasar de un estado a otro), depende única y exclusivamente del estado actual y no de la secuencia de estados anteriores. Existen diversos tipos de Procesos de Markov, por ejemplo las Cadenas de Markov, Los Procesos de Decisión de Markov y los Modelos Ocultos de Markov. Los Modelos Ocultos de Markov se caracterizan por el hecho de que los estados del sistema  que está siendo modelado no pueden ser observados directamente, pero otros eventos dependientes de los estados sí son observables. Cada estado (no observable) del sistema tiene una distribucion de probabilidad asociada a cada evento (observable) de dicho estado. El conjunto de todos los eventos observados puede ayudar a determinar cuáles fueron los estados que generaron dichos eventos.

En [22] se define un Modelo Oculto de Markov en el que se modelan los segmentos de individuales de calles como los estados no observables del sistema y todas las localizaciones de la trayectoria del vehículo como los eventos observados a partir de dichos estados. El objetivo del algoritmo es que dada una secuencia de eventos observados (las localizaciones) se encuentre la secuencia de estados (calles) que generaron dichas observaciones. Para realizar este proceso se deben tener en cuenta dos probabilidades, la probabilidad de emisión, que se define como la probabilidad de que un evento observado fue el resultado de un estado particular; y la probabilidad de transición, que se define como la probabilidad de pasar de un estado al siguiente. La probabilidad de emisión de cada localización disminuye a medida que aumenta la distancia entre la localización y la calle. Para la probabilidad de transición se compara la distancia entre dos localizaciones consecutivas con la distancia del camino más corto entre dos de sus correspondientes calles candidatas. Se utiliza el algoritmo de Viterbi calcular el camino óptimo a través del diagrama de estados del modelo. El algoritmo de Viterbi usa programación dinámica para encontrar rápidamente el camino que maximiza el producto de las probabilidades de emisión y de transición. El algoritmo está diseñado específicamente para funcionar de forma off-line pero puede ser adaptado utilizando una “ventana” de localizaciones para funcionar de forma on-line.

A diferencia de lógica convencional en la que una proposición puede tener dos valores, verdadero o falso, en la Lógica Difusa las proposiciones pueden ser “parcialmente” verdaderas, oscilando entre ser totalmente verdaderas o totalmente falsas. Por ejemplo, podemos decir que la velocidad del vehículo es alta, o que el tiempo de viaje es bajo, en lugar de usar cantidades exactas. Para inferir resultados se utilizan reglas difusas, como por ejemplo “si la velocidad del vehículo es alta y el tiempo de viaje es bajo, entonces la congestión en la calle es baja”. Las variables de entrada son la velocidad del vehículo y el tiempo de viaje y la variable de salida es la congestión. Consideremos por ejemplo la variable velocidad del vehículo, el valor de dicha variable se mide cuantitativamente en m/s y podemos clasificar la velocidad en tres posibles grupos: cero, baja y alta; cada posible valor de la velocidad corresponde en cierta medida a uno de estos tres grupos. La función  que determina el grado de pertenencia de un valor a cada uno de estos grupos se denomina “función de pertenencia”. El proceso completo consta de tres pasos: primero se convierten todas las variables de entrada a sus correspondientes valores difusos, a continuación se aplican las reglas de inferencia para obtener un resultado y finalment se convierten los resultados nuevamente a un valor concreto.

El algoritmo propuesto en [23] funciona de forma muy similar al propuesto en [21] pues también está divido en dos partes principales, el proceso inicial de selección y el proceso subsecuente de selección. A diferencia de [21] el proceso subsecuente de selección se divide en dos partes, el proceso de selección subsecuente a lo largo de un enlace y el proceso de selección subsecuente en una intersección. El proceso de selección inicial se aplica solo al principio del algoritmo para seleccionar la primera calle, a partir de ahí se aplica la  selección a lo largo de un enlace para determinar si el vehículo sigue sobre la calle seleccionada inicialmente, cuando se detecta una intersección se aplica el proceso de selección en una intersección para determinar si el vehículo ha hecho una maniobra de giro. Para el proceso inicial se tienen en cuenta la velocidad del vehículo, el error de dirección con respecto a la calle, la distancia perpendicular a la calle y el error horizontal del GPS; el resultado es la probabilidad de que la localización corresponda a una calle. El proceso de selección subsecuente a lo largo de un enlace tiene en cuenta la velocidad del vehículo, el diferencia de la dirección de la localización con respecto a la dirección de la localización anterior, la lectura del giroscopio y otras variables más; el resultado es la probabilidad de que la localización corresponda a la calle seleccionada anteriormente. Para el proceso de selección subsecuente en una intersección se aplican las mismas variables que para el proceso de selección inicial pero además se tienen en cuenta dos variables más, la conectividad de las calles con la calle seleccionada anteriormente y la diferencias entre la distancia recorrida por el vehículo y las distancias a las calles candidatas. Finalmente para cada paso se debe determinar la ubicación real del vehículo sobre la calle, esto se hace teniendo en cuenta la proyección de la localización sobre la calle, la velocidad del vehiculo y la direccion de desplazamiento. El algoritmo está especialmente diseñado para aplicaciones de tiempo real (on-line) con una alta frecuencia de muestreo.

Existen muchas otras técnicas avanzadas que se utilizan en conjunto con algunas de las técnicas descritas anteriormente ya sea como parte del algoritmo, como un proceso previo o como una forma de verificación del resultado. Entre estas técnicas podemos citar a la Distancia de Fréchet [24, 25], que determina la similaridad entre dos curvas y es utilizada en los algoritmos de curva a curva para determinar cual curva seleccionar, ademas también es generalmente utilizada para validar los resultados de los algoritmos. Otra técnica muy popular son los Filtros de Kalman y los Filtros Extendidos de Kalman [26], que generalmente se utilizan cuando existe una alta densidad de muestras y se utiliza como un paso previo en varios algoritmos para mejorar la calidad de las muestras iniciales.

\section{Algoritmos de Map Matching en Aplicaciones Móviles.}

Los algoritmos de map matching han sido utilizados en aplicaciones móviles para la planificación de rutas de viaje, asistentes de navegación personal, estimación de rutas de buses y principalmente para monitorear el estado del transito de forma cooperativa [27,  28]. Se han desarrollado además técnicas específicas para superar las limitaciones inherentes a las plataformas móviles (baja precisión de los sensores de localización, baja disponibilidad de energía de la batería) [9, 8].

En [28] se utiliza un algoritmo basado en un Modelo Oculto de Markov para determinar en tiempo real el camino que está recorriendo el usuario. Los datos de las localizaciones son enviados a un servidor central para determinar finalmente cuales son los caminos que tienen un mayor tiempo estimado de viaje. Los resultados son presentados tanto en la aplicación móvil instalada por el usuario como en una aplicación web. El algoritmo utiliza principalmente el GPS cuando está disponible pero también puede utilizar las localizaciones obtenidas mediante WiFi. Este trabajo demuestra que las localizaciones obtenidas mediante WiFi, si bien so menos precisas, son lo suficientemente confiables para realizar una estimación aceptable de los tiempos de viaje promedio en las calles y por consiguiente del estado del tránsito.

El sistema propuesto en [27] utiliza los sensores del teléfono, GPS, Wifi y acelerómetro para determinar si un usuario ha subido o no a un bus y a partir de ahí estimar el tiempo de llegada del bus a otras paradas, con el objetivo de reducir el tiempo de espera de los pasajeros. Se propone un algoritmo de clasificación de actividad basado en el acelerómetro para detectar cuando un usuario ha subido a un vehículo. Una vez que se detecta que el usuario está en un vehículo la aplicación empieza a rastrear al usuario usando GPS o WiFi y utiliza las localizaciones para determinar si el usuario está o no en un bus. Las localizaciones son enviadas a un servidor central. A diferencia de todos los algoritmos de map matching, en este caso en lugar de un mapa digital de calles se utiliza solamente el mapa de rutas de los buses. Se realiza un análisis espacial y temporal para determinar el bus al que ha subido el usuario.

En [9] se propone un esquema de bajo consumo de energía para la adquisición de datos de localizaciones. Se utiliza un algoritmo mejorado basado en Modelos Ocultos de Markov para determinar la ruta por la que ha viajado el vehículo. Para evitar el consumo innecesario de energía se adopta un método de muestreo adaptativo GPS que ajusta el período de muestreo del GPS basado en estado de movimiento actual del vehículo, sin embargo debido al algoritmo de map-matching utilizado, el tiempo de muestreo utilizado nunca es demasiado largo. Para mejorar el tiempo de ejecución del algoritmo se utiliza la información histórica de las calles seleccionadas anteriormente, así como la información topológica y las restricciones de velocidad de las calles. El algoritmo está diseñado para funcionar de forma on-line y el proceso de map-matching se realiza en el teléfono móvil.

En [8] describe un algoritmo que utiliza solamente información de las antenas de telefonía celular GMS y no hace uso de otros métodos para obtener localizaciones como el GPS y el WiFi que consumen mucha mas energia. La utilización de esta fuente de datos supone un desafío muy grande pues la precisión de las localizaciones es muy baja. Para procesar los datos se utiliza un algoritmo de dos pasadas basado en Modelos Ocultos de Markov que combina la información de las antenas GMS con información de otros sensores de bajo consumo de energía como el acelerómetro (para detectar el movimiento) y el giroscopio (para detectar giros). El algoritmo funciona de forma global y los datos son enviados y procesados en un servidor central.
