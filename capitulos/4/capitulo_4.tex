\chapter{Map Matching}
\label{cap:4}

Se conoce como \emph{Map Matching} (MM) al proceso de identificar la trayectoria seguida por un vehículo en una red de calles a partir de muestras recolectadas acerca de su ubicación. Este proceso es utilizado en una gran variedad de servicios y aplicaciones GIS, como la predicción de trayectorias de usuarios \cite{eisner2011algorithms}, sistemas de navegación en vehículos \cite{kim2001adaptive}, control del estado del tránsito \cite{thiagarajan2009vtrack}, estimación de llegada de buses \cite{thiagarajan2010cooperative}, entre otros. A continuación se define formalmente el proceso de MM y se presenta un análisis de los algoritmos de MM existentes en la actualidad.

\section{Definición del problema}

Para dar una definición formal del problema de MM deben tenerse en cuenta los siguientes conceptos:

\textbf{Muestra o Punto}: Una muestra o punto es el conjunto de todas las mediciones recolectadas por el vehículo en un instante dado. Estas mediciones incluyen la localización del vehículo, su dirección de desplazamiento, su velocidad, etc.

\textbf{Trayectoria}: Una trayectoria es una secuencia ordenada de muestras o puntos pertenecientes a un vehículo recolectados durante un viaje del mismo.

\textbf{Red de calles}: Una red de calles es un grafo dirigido que representa la forma y las propiedades del sistema de calles de un área geográfica en particular. Los vértices del grafo describen las intersecciones entre las calles y las aristas representan la forma y los atributos de las calles.

\textbf{Camino reconstruido}: Un camino reconstruido es una secuencia ordenada de calles conectadas entre sí a través de las cuáles se infiere que el vehículo pudo haber transitado.

En función a lo anterior, el problema de MM puede definirse de la siguiente forma: \emph{Dada una trayectoria T y una red de calles G, encontrar el camino real R que hace coincidir a T con su reconstrucción más realista sobre G.}

En la \Cref{fig:map-matching} se puede apreciar en rojo la trayectoria de un vehículo sobre una red de calles y en verde el posible camino reconstruido que se infiere el vehículo ha transitado.

\begin{figure}[h]
	\centering
	\input{capitulos/4/figuras/figura41.pdf_tex}
	\caption[Trayectoria y camino reconstruido mediante MM]{Ejemplo de trayectoria (rojo) y camino reconstruido (verde) mediante MM}
	\label{fig:map-matching} 
\end{figure}

\section{Clasificación de algoritmos de MM}

De acuerdo a la información y las técnicas utilizadas para su implementación los algoritmos de MM se pueden clasificar en: \begin{enumerate*}[1)]\item \emph{geométricos}, que utilizan sólo la información de posición y distancia entre las calles y puntos \cite{white2000some}, son sencillos de implementar y suelen utilizarse como paso inicial en la implementación de otros algoritmos; \item \emph{topológicos}, que incorporan información topológica de la red de calles \cite{lou2009map,yuan2010interactive,greenfeld2002matching,quddus2003general}, utilizan información sobre la conectividad, restricciones de giro, sentido y límite de velocidad de las calles; \item \emph{estadísticos}, que definen regiones de probabilidad alrededor de cada punto y analizan los tramos dentro de dichas regiones \cite{ochieng2009map}, estos análisis estadísticos también se utilizan como parte de otros algoritmos, y \item \emph{avanzados}, que combinan diversas técnicas geométricas, topológicas y estadísticas con otros conceptos tales como \emph{Filtros de Kalman}, \emph{Modelos Ocultos de Markov}, \emph{Lógica Difusa}, entre otros \cite{thiagarajan2009vtrack,quddus2006high,thiagarajan2011accurate,fang2011enacq}\end{enumerate*}.

De acuerdo al momento en el que se realiza el procesamiento de los datos, existen dos categorías: \begin{enumerate*}[1)] \item los algoritmos \emph{incrementales} u \emph{on-line}, que realizan el proceso de MM a medida que se van obteniendo nuevos puntos \cite{thiagarajan2009vtrack,thiagarajan2011accurate,greenfeld2002matching,quddus2003general,quddus2006high} y se utilizan en aplicaciones de tiempo real como asistentes personales de navegación, y \item los algoritmos \emph{globales} u \emph{off-line}, que realizan el proceso MM luego de que se han recolectado todos los puntos \cite{lou2009map,yuan2010interactive} y se utilizan en aplicaciones de análisis de tráfico o estudios sobre el comportamiento de usuarios\end{enumerate*}.

Dependiendo de la frecuencia de muestreo se pueden identificar dos categorías más: \begin{enumerate*}[1)] \item los algoritmos para \emph{alta frecuencia} o \emph{high-sampling}, que típicamente trabajan con intervalos de muestra en el rango de los pocos segundos y generalmente se ejecutan de forma \emph{on-line} \cite{greenfeld2002matching,quddus2003general,quddus2006high}, y \item los algoritmos para \emph{baja frecuencia} o \emph{low-sampling},  que funcionan para intervalos de muestreo de varios minutos y se ejecutan generalmente de forma \emph{off-line} \cite{lou2009map,yuan2010interactive}. \end{enumerate*} En general todos los algoritmos pueden ser alimentados con muestras de alta o baja frecuencia pero ciertos algoritmos dejan de ser efectivos a medida que disminuye la frecuencia de muestreo.

\subsection{Algoritmos geométricos}

Los algoritmos geométricos fueron los primeros en ser desarrollados, son los más simples y rápidos pero a la vez son los más propensos a errores. Estos algoritmos tienen en cuenta únicamente la posición de los puntos y de los vértices y aristas que conforman la red de calles \cite{quddus2007current}.

El algoritmo más sencillo se denomina de \emph{punto a punto} \cite{white2000some}, consiste en buscar para cada punto, el vértice dentro de la red de calles que sea el más cercano a dicho punto. Esta técnica es muy propensa a errores puesto que no se tiene en cuenta la conectividad entre los vértices seleccionados e implícitamente favorece a aquellas calles que tengan una mayor densidad de vértices, lo que puede llevar a resultados incorrectos. En la \Cref{fig:punto-a-punto} se puede apreciar cómo el punto $p_0$ es incorrectamente asociado al vértice $b_1$ cuando realmente está más próximo a la línea comprendida entre los vértices $a_0$ y $a_1$.

\begin{figure}[h]
	\centering
	\input{capitulos/4/figuras/figura42.pdf_tex}
	\caption{Error en el algoritmo de punto-a-punto}
	\label{fig:punto-a-punto} 
\end{figure}

Otra técnica utilizada se denomina \emph{punto a curva} \cite{white2000some}, y consiste en buscar para cada punto, la calle (curva) del mapa más cercana a dicho punto. Como tampoco se tiene en cuenta la conectividad entre las calles, se pueden tener los mismos errores que con la técnica anterior. Esta técnica no es efectiva en zonas urbanas que tienen una mayor densidad de calles debido al margen de error de las localizaciones. En la \Cref{fig:punto-a-curva} se puede apreciar cómo el punto $p_4$ es incorrectamente asignado a la línea comprendida entre los vértices $b_0$ y $a_0$ debido a que la distancia a esta línea es más corta que la distancia a la línea comprendida entre $a_0$ y $a_1$.

\begin{figure}[h]
	\centering
	\input{capitulos/4/figuras/figura43.pdf_tex}
	\caption{Error en el algoritmo de punto-a-curva}
	\label{fig:punto-a-curva} 
\end{figure}

La última técnica dentro de esta categoría es conocida como \emph{curva a curva} \cite{white2000some}, y consiste en  utilizar la técnica de punto a punto para identificar un vértice candidato para un punto, luego se seleccionan todos los tramos que se originan en dicho vértice y se calcula la distancia entre cada uno de estos tramos y la recta comprendida entre el punto actual y un punto siguiente. El tramo del mapa que resulta ser el más cercano es seleccionado como el tramo real que recorre el vehículo. Un tramo puede estar compuesto por una o más secciones de una calle. La forma específica en que se definen estos tramos y la función que calcula la distancia puede variar entre implementaciones. En la \Cref{fig:curva-a-curva} el vértice $o$ es el vértice candidato para el punto $p1$, a partir de $o$ se originan los tramos $A$, comprendido entre los puntos $o$ y $a_1$, $B$, comprendido entre los puntos $o$ y $b_1$, y $C$, entre los puntos $o$ y $c_0$. El tramo más cercano a la recta comprendida entre $p_0$ y $p_1$ es $A$. 

\begin{figure}[h]
	\centering
	\input{capitulos/4/figuras/figura45.pdf_tex}
	\caption{Algoritmo de curva-a-curva}
	\label{fig:curva-a-curva} 
\end{figure}

\subsection{Algoritmos topológicos}

Se conoce como \emph{topología} a la relación entre las distintas formas geométricas, como ser puntos, líneas y polígonos. Entre estos elementos pueden definirse relaciones de adyacencia y conectividad. Los mapas de calles se representan generalmente como puntos y líneas, donde las líneas representan secciones de calles y los puntos representan intersecciones entre las calles. Por ejemplo, los mapas digitales cuentan con relaciones topológicas tales como límites de velocidad, restricciones de giro y sentido de circulación. Así, los algoritmos de MM que incorporan este tipo de información en la reconstrucción del camino de un vehículo se conocen como algoritmos topológicos \cite{quddus2007current}.

El algoritmo topológico desarrollado en \cite{greenfeld2002matching} utiliza distintos criterios de similaridad para determinar la mejor calle candidata para cada punto. Los criterios utilizados son la similaridad en la orientación entre dos puntos consecutivos y la calle candidata, la proximidad entre la localización del punto y la calle candidata y el tamaño del ángulo comprendido entre la dirección de desplazamiento del vehículo (\emph{bearing}) y la dirección de la calle candidata. Para cada calle candidata se realiza una suma ponderada de los criterios y se elige como mejor candidata a aquella con la mayor suma.

El mismo procedimiento es utilizado en \cite{quddus2003general} pero se agrega información adicional como la velocidad del vehículo y la posición relativa del punto con respecto al vértice más cercano. En este algoritmo el proceso para determinar si el vehículo sigue o no en la calle seleccionada anteriormente se aplica cada vez que el vehículo pasa por una intersección de calles. Ambos métodos requieren que se identifique correctamente la primera calle, para luego ir eligiendo las calles que tienen conexión con la misma. Un fallo en la elección inicial puede llevar a resultados incorrectos.

El algoritmo desarrollado en \cite{lou2009map}, conocido como \emph{ST-Matching}, utiliza información geográfica y topológica para asignar un valor numérico a cada camino posible y luego selecciona el camino con el mayor valor. El algoritmo utiliza la técnica de punto a curva para determinar el conjunto de puntos candidatos para cada punto de la trayectoria. Existe un camino entre cada punto candidato y todos los puntos candidatos del punto siguiente, formando así todos los posibles caminos del vehículo. Para asignar el valor numérico a cada camino se tienen en cuenta dos tipos de análisis: \begin{enumerate*}[a)] \item el \emph{análisis espacial}, en el que se calcula una probabilidad de observación para cada punto candidato y una probabilidad de transmisión entre cada par de puntos candidatos consecutivos, y \item el \emph{análisis temporal}, en el que se calcula la similaridad entre la velocidad promedio entre dos puntos candidatos y las restricciones de velocidad del camino comprendido entre los puntos.\end{enumerate*} Otros trabajos posteriores han agregado diversas mejoras al algoritmo original, como un proceso de detección de localizaciones inválidas en  \cite{sakic2012map} y, la normalización del cálculo de la probabilidad de transmisión y la prevención de bucles en el camino final obtenido que se presenta en  \cite{budigm2012algorithm}.

En \cite{yuan2010interactive} se propone un algoritmo basado en ST-Matching que incorpora el concepto de \emph{voto interactivo} para modelar la influencia mutua que tienen entre sí todos los puntos de la trayectoria del vehículo (a mayor distancia entre candidatos, menor la influencia). Para cada punto candidato existe un conjunto de caminos posibles que pasan por él, el objetivo es determinar cuál de los caminos es el óptimo para cada punto candidato, luego cada punto candidato vota por su “mejor camino” y finalmente se selecciona el camino óptimo global de acuerdo al resultado de esta votación.

Los algoritomos presentados en \cite{lou2009map} y \cite{yuan2010interactive}, están específicamente diseñados para trabajar de forma \emph{off-line} y con una baja frecuencia de muestreo. En \cite{lou2009map} se  menciona que es posible utilizar el algoritmo en aplicaciones \emph{on-line} definiendo una “ventana” de localizaciones para las cuales se realiza el procedimiento de MM. En \cite{sakic2012map} se realizaron varias adaptaciones para utilizar el algoritmo en una aplicación de tiempo real, definiendo un tamaño de ventana en una localización, para convertir el algoritmo global en un algoritmo incremental.

\subsection{Algoritmos estadísticos}

Los algoritmos estadísticos, también conocidos como probabilísticos, son aquellos que definen regiones de “confiabilidad” alrededor de las localizaciones y seleccionan una calle candidata de entre todas aquellas que están dentro de esta región \cite{quddus2007current}. Para determinar la región de confianza se tienen en cuenta los posibles errores originados por los sensores utilizados para obtener la localización y los errores en el mapa digital. Estos algoritmos también pueden utilizar información geométrica y topológica de la red de calles.

El algoritmo desarrollado en \cite{ochieng2009map} toma en cuenta diversas fuentes de error asociadas con los sensores de localización, la trayectoria anterior del vehículo, la información topológica de las calles (conectividad y orientación de las calles), e información sobre la velocidad y la dirección del vehículo. Este algoritmo está específicamente diseñado para aplicaciones con requerimientos de tiempo real y con una alta frecuencia de muestreo. El algoritmo se divide en dos partes, \begin{enumerate*}[a)]
\item el proceso de inicial de selección (Initial Matching Process), y \item el proceso de selección subsecuente (Subsequent Matching Process)\end{enumerate*}. 

En el \emph{proceso de selección inicial} se define alrededor de la primera localización una región de confiabilidad rectangular o elíptica, si dentro de esta región no existe ningún candidato se asume que el vehículo está fuera de la red de calles, si existe más de un candidato se utiliza información sobre la conectividad de los candidatos y la dirección de desplazamiento del vehículo para determinar el candidato más apropiado. El \emph{proceso de selección  subsecuente} es utilizado para determinar si en la siguiente localización el vehículo sigue viajando o no por la misma calle, para ello se intenta identificar si el vehículo ha hecho alguna maniobra de giro o si está atravesando una intersección de calles y en caso de que se detecte alguna de estas dos condiciones se vuelve a realizar el proceso de selección inicial para determinar la siguiente calle sobre la que esta viajando el vehículo.

\subsection{Algoritmos avanzados}

Los algoritmos avanzados son aquellos que combinan diversas técnicas topológicas, geométricas y estadísticas con otras técnicas avanzadas \cite{quddus2007current}. Las técnicas más utilizadas en los algoritmos implementados en los últimos años son la Lógica Difusa y los Modelos Ocultos de Markov.

Se conocen como \emph{Procesos de Markov} a aquellos procesos estocásticos que cumplen con la condición de que la probabilidad de transición entre dos estados depende única y exclusivamente del estado actual y no de la secuencia de estados anteriores. Existen diversos tipos de Procesos de Markov, por ejemplo las Cadenas de Markov, Los Procesos de Decisión de Markov y los Modelos Ocultos de Markov. Los Modelos Ocultos de Markov se caracterizan por el hecho de que los estados del sistema que está siendo modelado no pueden ser observados directamente, pero otros eventos dependientes de los estados sí son observables \cite{blunsom2004hidden}. Cada estado no observable del sistema tiene una distribución de probabilidad asociada a cada evento observable de dicho estado, denominada probabilidad de emisión. El conjunto de todos los eventos observados puede ayudar a determinar cuáles fueron los estados que generaron dichos eventos. La \Cref{fig:modelo-markov} muestra los parámetros de un Modelo Oculto de Markov.

\begin{figure}[h]
	\centering
	\input{capitulos/4/figuras/figura44.pdf_tex}
	\captionsetup{singlelinecheck=off}
	\caption[Parámetros de un modelo oculto de Markov]{
	Parámetros probabilísticos de un modelo oculto de Markov: \begin{itemize}
	\item $X$: estados no observables
	\item $y$: eventos observables
	\item $a$: probabilidad de transición
	\item $b$: probabilidad de emisión
	\end{itemize}
	}
	\label{fig:modelo-markov} 
\end{figure}

En \cite{newson2009hidden} se define un Modelo Oculto de Markov en el que se modelan los segmentos individuales de calles como los estados no observables del sistema y todos los puntos de la trayectoria del vehículo como los eventos observados a partir de dichos estados. El objetivo del algoritmo es que dada una secuencia de eventos observados (los puntos) se encuentre la secuencia de estados (calles) que generaron dichas observaciones. La probabilidad de emisión de cada punto disminuye a medida que aumenta la distancia entre el punto y la calle. Para obtener la probabilidad de transición se compara la distancia entre dos puntos consecutivos con la distancia del camino más corto entre dos de sus correspondientes calles candidatas. Se utiliza el algoritmo de Viterbi para calcular el camino óptimo a través del diagrama de estados del modelo. El algoritmo de Viterbi utiliza la técnica de Programación Dinámica para encontrar rápidamente el camino que maximiza el producto de las probabilidades de emisión y de transición \cite{forney1973viterbi}. Este proceso de MM está diseñado específicamente para funcionar de forma \emph{off-line} pero puede ser adaptado utilizando una ventana de localizaciones para funcionar de forma \emph{on-line}.

Otra técnica utilizada en algoritmos de MM es la Lógica Difusa. A diferencia de lógica convencional en la que una proposición puede tener dos valores, verdadero o falso, en la Lógica Difusa las proposiciones pueden ser \emph{parcialmente verdaderas}, oscilando entre ser \emph{totalmente verdaderas} o \emph{totalmente falsas}. Por ejemplo, podemos decir que la velocidad del vehículo es \emph{alta}, o que el tiempo de viaje es \emph{bajo}, en lugar de usar cantidades exactas. Para inferir resultados se utilizan reglas difusas, como por ejemplo: “Si la velocidad del vehículo es \emph{alta} y el tiempo de viaje es \emph{bajo}, entonces la congestión en la calle es \emph{baja}”. Las variables de entrada son la \emph{velocidad} del vehículo y el \emph{tiempo} de viaje, y la variable de salida es la \emph{congestión}. Consideremos por ejemplo la variable \emph{velocidad} del vehículo, el valor de dicha variable se mide cuantitativamente en $m/s$ y podemos clasificar la velocidad en tres posibles grupos: \emph{cero}, \emph{baja} y \emph{alta}; cada posible valor de la velocidad corresponde en cierta medida a uno de estos tres grupos. La función que determina el grado de pertenencia de un valor a cada uno de estos grupos se denomina \emph{función de pertenencia}. El proceso completo consta de tres pasos: \begin{enumerate*}[1)]
\item primero se convierten todas las variables de entrada a sus correspondientes valores difusos, 
\item a continuación se aplican las reglas de inferencia para obtener un resultado, y
\item finalmente se convierten los resultados nuevamente a un valor concreto
\end{enumerate*}. Una explicación más detallada se puede encontrar en \cite{zadeh1988fuzzy}

El algoritmo propuesto en \cite{quddus2006high} funciona de forma muy similar al propuesto en \cite{ochieng2009map} pues también está divido en dos partes principales: el \emph{proceso de selección inicial} y el \emph{proceso de selección subsecuente}, que se divide a su vez nuevamente en dos partes: \emph{la selección a lo largo de un enlace} y  \emph{la selección en una intersección}. El \emph{proceso inicial} se aplica para seleccionar la primera calle, a partir de ahí se aplica \emph{la selección a lo largo de un enlace} para determinar si el vehículo sigue sobre la misma calle, cuando se detecta una intersección se aplica \emph{la selección en una intersección} para determinar si el vehículo ha hecho una maniobra de giro. Para el \emph{proceso inicial} se tienen en cuenta la velocidad, la dirección, la distancia perpendicular a la calle y el error horizontal del GPS. El \emph{proceso de selección subsecuente a lo largo de un enlace} tiene en cuenta la velocidad, la dirección, la lectura del giroscopio y otras variables más. Para el \emph{proceso de selección subsecuente en una intersección} se aplican dos variables más, la conectividad de las calles y la distancia recorrida por el vehículo. Finalmente, para cada paso se determina la ubicación del vehículo sobre la calle teniendo en cuenta la proyección del punto sobre la calle, la velocidad y dirección del vehículo. El algoritmo está especialmente diseñado para aplicaciones de tiempo real (\emph{on-line}) con una alta frecuencia de muestreo.

Existen otras técnicas avanzadas que se utilizan en conjunto con algunas de las técnicas descritas anteriormente ya sea como parte del algoritmo, como un proceso previo o como una forma de verificación del resultado. Entre estas técnicas podemos citar a la Distancia de Fréchet \cite{chen2011approximate,eisner2011algorithms}, que determina la similaridad entre dos curvas y es utilizada en los algoritmos de curva a curva para determinar la curva a seleccionar. La Distancia de Fréchet también es generalmente utilizada para validar los resultados de los algoritmos. Otra técnica muy popular son los Filtros de Kalman y los Filtros Extendidos de Kalman \cite{kim2001adaptive}, que generalmente se utilizan cuando existe una alta densidad de muestras y se desea realizar un paso previo para mejorar su calidad.

\section{MM en aplicaciones móviles.}

Los algoritmos de MM han sido utilizados en aplicaciones móviles para la planificación de rutas de viaje, asistentes de navegación personal, estimación de rutas de buses y para aproximar el estado del tránsito de forma cooperativa \cite{thiagarajan2010cooperative, thiagarajan2009vtrack}. Se han desarrollado además técnicas específicas para superar las limitaciones inherentes a las plataformas móviles como la baja precisión de los sensores de localización  \cite{thiagarajan2011accurate, fang2011enacq}.

En \cite{thiagarajan2009vtrack} se utiliza un algoritmo basado en un Modelo Oculto de Markov para determinar en tiempo real el camino que está recorriendo el usuario. Los datos de las localizaciones son enviados a un servidor central para determinar finalmente cuales son los caminos que tienen un mayor tiempo estimado de viaje. El algoritmo utiliza principalmente el GPS cuando está disponible pero también puede utilizar las localizaciones obtenidas mediante WiFi. Este trabajo demuestra que las localizaciones obtenidas mediante WiFi, si bien son menos precisas, son lo suficientemente confiables para realizar una estimación aceptable de los tiempos de viaje promedio en las calles y por consiguiente del estado del tránsito.

El sistema propuesto en \cite{thiagarajan2010cooperative} utiliza los sensores del teléfono, tales como de GPS, Wifi y acelerómetro para determinar si un usuario ha subido o no a un bus y a partir de ahí estimar el tiempo de llegada del bus a otras paradas, con el objetivo de reducir el tiempo de espera de los pasajeros. Se utiliza un algoritmo de curva a curva para seleccionar la ruta de buses que se aproxima más a la secuencia de localizaciones obtenidas por GPS, luego se realiza un post-procesamiento para identificar si el vehículo es o no un bus de transporte público. En este caso en lugar de un mapa digital de calles se utiliza solamente el mapa de rutas de los buses.

En \cite{fang2011enacq} se utiliza un algoritmo de MM mejorado basado en Modelos Ocultos de Markov para determinar la ruta por la que ha viajado el vehículo. La toma de localizaciones se realiza mediante un método de muestreo adaptativo basado en GPS. El algoritmo además utiliza información histórica de los caminos recorridos anteriormente  para mejorar su tiempo de ejecución, así como también información topológica y restricciones de velocidad de las calles para inferir los recorridos. Este algoritmo está diseñado para funcionar de forma \emph{on-line} y el proceso de MM se realiza en el teléfono móvil.

En \cite{thiagarajan2011accurate} se describe un algoritmo que utiliza solamente información de las antenas de telefonía celular GSM para obtener localizaciones. Para procesar los datos se utiliza un algoritmo de MM de dos pasadas basado en Modelos Ocultos de Markov. El algoritmo funciona de forma global y los datos son enviados y procesados en un servidor central.
