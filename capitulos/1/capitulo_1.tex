\chapter{Introducción}

\section{Motivación}

La congestión de tránsito es uno de los problemas más serios que enfrentan las zonas urbanas del mundo hoy en día, tanto en países desarrollados como en países en vías de desarrollo, el constante aumento del parque de automóviles, el alto uso de vehículos de transporte privado y la falta de planificación de las ciudades son factores que deterioran la calidad de vida de los ciudadanos.

El fuerte impacto negativo de la congestión exige un esfuerzo multidisciplinario que apunte a mejorar la situación actual mediante el diseño de políticas públicas de mejoramiento de la infraestructura, la implementación de sistemas de monitoreo y control de tránsito eficientes, la mejora del sistema de transporte público y la mejora en la educación vial de la población.

Los países menos desarrollados no cuentan con sistemas integrados de control de tránsito debido al alto costo de inversión requerido, tampoco existen servicios proveídos por empresas privadas. Esta situación dificulta la utilización eficiente de las escasas y mal planificadas vías de tránsito existentes, es por eso que se hace evidente la necesidad de recolectar y analizar información sobre el tránsito de forma barata, sencilla y que requiera una mínima inversión.

En este trabajo se propone la implementación de un sistema que permitirá monitorear las condiciones del tránsito utilizando información proveída por dispositivo móviles en tiempo real. Para ello se estudiará el estado del arte de las técnicas de recolección, análisis y distribución de información de tráfico y se seleccionarán los mecanismos de monitoreo y análisis más adecuados a las necesidades y limitaciones de nuestro medio. Dicha información quedará disponible de forma pública y podrá ser utilizada en trabajos futuros para desarrollar soluciones que ayuden a mitigar la problemática de la congestión vehicular.

\section{Objetivos}

El principal objetivo de este proyecto es construir un sistema que permita recolectar y procesar información del estado del tránsito vehicular en tiempo real a través de dispositivos móviles para ofrecer funcionalidades que ayuden al control y/o reducción de la congestión de tránsito.

Entre los objetivos específicos se puede citar:

\begin{itemize}

\item Estudiar el estado del arte de las técnicas de recolección de datos de tránsito con un enfoque particular en el área de Floating Car Data.

\item Estudiar el estado del arte en técnicas de Map Matching utilizadas para procesar la información obtenida.

\item Estudiar el estado del arte de técnicas de Distribución de la Información entre dispositivos móviles.

\item Diseñar la arquitectura para un sistema que utilice las técnicas estudiadas de modo a recolectar, analizar, distribuir y utilizar la información del estado del tránsito vehicular.

\item Implementar una aplicación móvil que se utilizará para recolectar datos de ubicación de los usuarios, distribuir la información de tránsito procesada y brindar soporte a los usuarios.

\item Implementar un sistema centralizado capaz de recibir los datos de ubicación de los usuarios para determinar el estado del tránsito vehicular.

\end{itemize}

\section{Organización}

Este trabajo está dividido en 5 capítulos. El capítulo 1 describe la motivación para realizar el trabajo, brinda una visión general de los desafíos existentes y define los objetivos que se desean alcanzar.

%TODO corregir, cambio la numeración de capítulos.

En el capítulo 2 se hace un extenso estudio del estado del arte de las técnicas de recolección, análisis y distribución de los datos necesarios para la implementación del sistema, con un especial énfasis en las técnicas de Floating Car Data para la recolección de datos y de Map Matching para el análisis de los mismos.

El capítulo 3  describe el problema y se explica detalladamente la arquitectura de la solución propuesta, las técnicas seleccionadas para la recolección y el análisis de datos y los algoritmos utilizados en casa paso del proceso.

En el capítulo 4 se presenta un análisis de las pruebas realizadas para verificar el funcionamiento del sistema y muestra los resultados obtenidos. Finalmente en el capítulo 5 se presentan las conclusiones y se proponen los posibles trabajos futuros.
