\chapter{Sistemas de Información Geográfica}
\label{cap:2}

Se conocen como Sistemas de Información Geográfica (\emph{Geographic Information Systems} - GIS) a aquellos sistemas basados en computadoras diseñados para la recolección, mantenimiento, análisis y distribución de datos e información geográfica \cite{bolstad2005gis}. Un GIS es un sistema de hardware, software, personas, datos, organización y mecanismos institucionales para recoger, almacenar, analizar y difundir información geográfica \cite{longley2005geographic}.

Los GIS ayudan en el análisis y la toma de decisiones basadas en información geográfica y son aplicables en diversas áreas de la actividad humana como la planificación para el desarrollo de zonas urbanas y rurales, la delimitación de propiedades y el cobro de  impuestos, el análisis del flujo del tráfico en ciudades, etc. En este capítulo se explican los conceptos fundamentales que sirven de base para el desarrollo del GIS propuesto en este trabajo.

\section{Datums y Coordenadas Geográficas}

La tarea fundamental de cualquier GIS es conocer e identificar de forma precisa la ubicación de los objetos de interés dentro de un mapa. En esta sección describimos los principios básicos que permiten definir esta ubicación y representarla en el mapa.

\subsection{Sistemas de Coordenadas}
\label{sistema_de_coordenadas}

El primer paso para definir la ubicación de los objetos es establecer el sistema de coordenadas a utilizar. Se denominan como \emph{coordenadas} al conjunto de números que identifican de forma única a una posición dentro de un sistema de coordenadas. Estos números son representados generalmente como pares $x$,$y$ en un plano cartesiano, o como valores de \emph{latitud} y \emph{longitud} en un sistema de coordenadas geográficas. Para definir el sistema de coordenadas geográficas se tienen en cuenta dos elementos principales: \begin{enumerate*}[1)] \item el tamaño y la forma de la Tierra y \item un conjunto de elementos de referencia a partir de los cuales se puede determinar la ubicación de cualquier otro objeto en la superficie de la Tierra\end{enumerate*}.

La Tierra es modelada como un elipsoide achatado hacia los polos cuyo semieje mayor corresponde al radio en la dirección del ecuador y cuyo semieje menor corresponde al radio en la dirección de los polos. Los polos están definidos por el eje de rotación del elipsoide y el ecuador se define como el círculo a mitad de camino entre los dos polos, en un ángulo recto con respecto al eje polar, y que cubre la dimensión más ancha de la elipsoide (Ver \Cref{fig:elipsoide}).

\begin{figure}[h]
	\centering
	\input{capitulos/2/figuras/figura1.pdf_tex}
	\caption[Elipsoide que modela la superficie de la Tierra]{Elipsoide utilizado para modelar la superficie del planeta Tierra}
	\label{fig:elipsoide} 
\end{figure}

Los sistemas de coordenadas geográficas se componen de la \emph{latitud}, que varía de norte a sur, y la \emph{longitud}, que varía de este a oeste. Las líneas de longitud constante son denominadas \emph{meridianos}, y las líneas de latitud constante son denominadas \emph{paralelos}. Los paralelos son líneas paralelas entre sí en dirección este-oeste alrededor de la Tierra. Los meridianos son líneas que van de sur a norte y convergen en los polos. Por convención, el ecuador se toma como la latitud cero y las latitudes al sur y al norte varían en un rango de hasta 90°. Se toma como longitud cero al meridiano que atraviesa el Observatorio Real de Greenwich, en Inglaterra y las longitudes al este u oeste se especifican como ángulos de giro con respecto al meridiano de Greenwich, variando de -180° (oeste) a 180° (este) (Ver \Cref{fig:coordenadas}).

\begin{figure}[h]
	\centering
	\input{capitulos/2/figuras/figura2.pdf_tex}
	\caption[Sistema de coordenadas de latitud y longitud]{Sistema de coordenadas de latitud y longitud}
	\label{fig:coordenadas} 
\end{figure}

\subsection{Datums geodésicos}

%TODO explicar qué es Geodesia.

Con el sistema de coordenadas definido se procede a estimar las longitudes y latitudes de todas las demás ubicaciones a través de mediciones. Mediante estas estimaciones se establecen un conjunto de puntos de referencia para los cuales la latitud y la longitud han sido precisamente determinadas. Todas las demás coordenadas se miden en referencia a este conjunto de puntos. El conjunto de puntos resultante se conoce como \emph{datum geodésico}.  Un datum geodésico consta de dos componentes principales. El primer componente es un elipsoide con un sistema de coordenadas esféricas o cartesianas de tres dimensiones y un origen. La segunda parte consiste en un conjunto de puntos y líneas cuyas coordenadas han sido medidas cuidadosamente utilizando los mejores métodos y equipos \cite{bolstad2005gis}.

A lo largo de la historia diferentes estudios han determinado diversos valores para los radios y el centro del elipsoide. Debido a errores de medición, diferencias en los métodos de cálculo y a que la tierra no es un elipsoide perfecto, diferentes estimaciones alrededor del mundo tienen ligeramente diferentes orígenes, ejes de orientación y radios. Estas diferencias aunque pequeñas, generalmente resultan en estimaciones bastante diferentes para las coordenadas de una misma localización, dependiendo del elipsoide utilizado. En la actualidad las mediciones globales y la avanzada capacidad computacional permiten estimar elipsoides aplicables de forma global.

El primer datum utilizado ampliamente en Norteamérica fue el \emph{North American Datum} de 1927 (NAD27), su sucesor fue el \emph{North American Datum} de 1983 (NAD83), las mejores estimaciones de las localizaciones de los puntos cambiaron en hasta 200 metros entre estos dos datums. En la \Cref{fig:datum-latitud} se puede apreciar el desplazamiento de la latitud entre los datums NAD83 y NAD27 y en la \Cref{fig:datum-longitud} se visualiza la diferencia de longitud entre los mismos. El \emph{World Geodetic System} de 1984 (WGS84) es un conjunto de datums globales desarrollado y utilizado principalmente por el Departamento de Defensa de los Estados Unidos y hoy en día es ampliamente aceptado y utilizado en todo el mundo \cite{longley2005geographic}.

\begin{figure}[h]
	\centering
	\input{capitulos/2/figuras/figura3.pdf_tex}
	\caption[Desplazamiento de la latitud entre NAD83 y NAD27]{Desplazamiento de la latitud de los datums en metros, NAD83 menos NAD27}
	\label{fig:datum-latitud} 
\end{figure}

\begin{figure}[h]
	\centering
	\input{capitulos/2/figuras/figura4.pdf_tex}
	\caption[Desplazamiento de la longitud entre NAD83 y NAD27]{Desplazamiento de la longitud de los datums en metros, NAD83 menos NAD27}
	\label{fig:datum-longitud} 
\end{figure}

%\subsection{Proyecciones}

%Una \emph{proyección} es una representación sistemática de las posiciones de la superficie curva de la Tierra sobre una superficie plana de un mapa. El mapa resultante sufre una distorsión debido a la traducción de una superficie curva a una plana. La proyección puede definirse por lo tanto como la transformación de una posición en la superficie de la Tierra, identificada por su latitud y longitud $(\varphi,\lambda)$ a una posición en coordenadas cartesianas $(x,y)$. Cualquier proyección puede representarse como un par de funciones matemáticas: \begin{displaymath} x=f(\varphi,\lambda );\quad y=g(\varphi,\lambda ) \end{displaymath}

%Las proyecciones se pueden clasificar de acuerdo a su forma, en \emph{cónicas}, \emph{cilíndricas} o \emph{planas} (ver \Cref{fig:tipos-proyecciones}) y de acuerdo a su orientación en \emph{ecuatoriales}, cuando el eje de la proyección coincide con los polos, o \emph{transversales}, cuando coincide con el ecuador.

%\begin{figure}[h]
%	\centering
%	\input{capitulos/2/figuras/figura5.pdf_tex}
%	\caption[Tipos de proyecciones por forma]{Tipos de proyecciones: a) cilíndrica, b) cónica y c) plana}
%	\label{fig:tipos-proyecciones} 
%\end{figure}

%Algunas de las proyecciones más importantes son la Proyección Cilíndrica Equidistante; la Proyección Conforme de Lambert (cónica), la Proyección Transversa de Mercator (cilíndrica) y el Sistema de Coordenadas Universal Transversal de Mercator (UTM por sus siglas en Inglés), que está basada en la Proyección Normal de Mercator pero en lugar de ser tangente al ecuador es tangente a un meridiano.

\section{Modelos de datos}

Los fenómenos geográficos requieren de dos características para representar el mundo real; \emph{qué} se está representando, y \emph{dónde} está. Para lo primero comúnmente utilizamos conceptos como `ciudad', `calle', `río', etc. Los modelos de datos geográficos son los equivalentes formales de estos modelos conceptuales y definen las estructuras de datos abstractas que serán almacenadas en las bases de datos espaciales y manipuladas por el GIS \cite{burrough1998principles}.

\subsection{Modelo de Datos Vectorial}

En un modelo vectorial cada objeto del mundo real es representado como uno de tres posibles tipos de datos geométricos: \emph{puntos}, \emph{líneas} o \emph{polígonos} (ver \Cref{fig:modelo-vectorial}). Un punto se representa de forma única por sus coordenadas (ciudades, puestos de salud, comisarías), las líneas se representan como una lista ordenada de coordenadas correspondientes a sus vértices (rutas, ríos y arroyos) y los polígonos se representan como uno o más segmentos de líneas que se cierran para formar una figura delimitada (propiedades rurales, lagos, distritos). Las coordenadas que definen la geometría de cada objeto pueden tener 2, 3, 4 o más dimensiones. En algunos modelos de datos las figuras geométricas pueden ser representadas por curvas definidas por una función matemática (ej. curvas de Biézer).

\begin{figure}[h]
	\centering
	\input{capitulos/2/figuras/figura7.pdf_tex}
	\caption{Modelo de datos vectorial}
	\label{fig:modelo-vectorial} 
\end{figure}

Objetos del mismo tipo (ej. calles) normalmente son guardados en una tabla de base de datos en la que cada objeto ocupa una fila de la tabla y cada atributo del objeto corresponde a una columna de la tabla. En un GIS también se puede almacenar información topológica sobre los objetos, por ejemplo las relaciones de adyacencia y conectividad entre las calles o su sentido de circulación y su límite de velocidad.

\subsection{Modelo de Datos Raster}

%TODO agregar un footenote a la palabra "raster" con un link a Wikipedia por ej.

En una representación \emph{raster} el espacio es dividido en una colección de celdas rectangulares, todas las variaciones geográficas son expresadas asignando propiedades o atributos a estas celdas. Este tipo de representación requiere que la superficie curva de la tierra sea proyectada en una superficie plana. Debido a que esto genera distorsiones, las celdas en una representación \emph{raster} no son perfectamente iguales en forma o área a la superficie que representan. Cuando se usa este formato, toda la información acerca de las variaciones dentro de una celda es perdida y se asigna un único valor a toda la celda (Ver \Cref{fig:modelo-raster}).

\begin{figure}[h]
	\centering
	\input{capitulos/2/figuras/figura8.pdf_tex}
	\caption{Modelo de datos raster}
	\label{fig:modelo-raster} 
\end{figure}

Las celdas pueden contener atributos basados en diversos esquemas de codificación. Por ejemplo, un esquema de codificación simple puede asignar un valor binario, cero o uno, a cada celda para representar la presencia o ausencia de forestación en dicha celda. Múltiples atributos pueden ser guardados para cada celda en una tabla en la que cada columna es un atributo de la celda y cada fila representa a una celda en particular. Los datos usualmente son guardados en archivos comprimidos o en bases de datos relacionales.

\subsection{Modelos de Datos de Red}

El modelo de red es un tipo especial de modelo topológico que puede ser utilizado para modelar el flujo de bienes y servicios. En un GIS las redes son modeladas como puntos o nodos (ej. intersecciones de calles, fusibles, interruptores, válvulas de agua) y líneas (calles, líneas de transmisión y tuberías). Las relaciones topológicas de la red definen cómo las líneas se conectan entre sí en los nodos y también pueden definir reglas sobre cómo los flujos pueden moverse a través de una red.

Por ejemplo, en una red de calles que está compuesta por un conjunto de nodos (intersecciones de las calles) y líneas (las calles), así como las relaciones topológicas entre las mismas (restricciones de giro en los nodos, sentido de las calles, límites de velocidad), la información topológica permite analizar el flujo de tráfico a través de la red y determinar los puntos más congestionados o calcular caminos más cortos de un punto de la red a otro.

\subsection{Red Irregular de Triángulos (TIN)}

Los modelos de datos geográficos descritos hasta ahora se utilizan para representar datos en una o dos dimensiones. Para representar superficies tridimensionales en un GIS se pueden utilizar modelos \emph{raster} o redes irregulares de triángulos (\emph{Triangulated Irregular Network} - TIN). En los modelos \emph{raster} cada celda almacena la altura de la superficie en una localización dada. En una estructura TIN se representa una superficie como una serie de elementos triangulares contiguos que no se superponen entre sí.

Un TIN es una estructura topológica que almacena información acerca de los nodos que conforman cada triángulo y sus triángulos vecinos. El triángulo está representado por una secuencia de tres nodos, cada nodo está definido por sus coordenadas $x$, $y$, $z$ (Ver \Cref{fig:modelo-tin}). Los triángulos tienen como atributos su pendiente, orientación y área, los vértices tienen atributos de elevación y los bordes tienen como atributos su pendiente y dirección. Al igual que en los demás modelos de datos, cada triángulo puede tener otros datos o atributos asociados.

\begin{figure}[h]
	\centering
	\input{capitulos/2/figuras/figura9.pdf_tex}
	\caption{Red irregular de triángulos}
	\label{fig:modelo-tin} 
\end{figure}

\section{Bases de Datos Espaciales}

Una base de datos espacial es aquella en la que están definidos tipos de datos especiales para representar objetos geométricos y permite guardar estos objetos, que usualmente son de origen geográfico, en tablas regulares de una base de datos. Además provee funciones e índices para consultar y manipular estos datos \cite{obe2011postgis}. Las bases de datos espaciales no necesitan ser relacionales pero muchas de las más populares lo son.

En las bases de datos espaciales se definen tipos de datos que representan puntos, líneas y polígonos a partir de los cuales se pueden construir otros tipos de datos más complejos y representar objetos del mundo real. Estas bases de datos están especialmente diseñadas para representar información en dos dimensiones en formato vectorial, pero también pueden almacenar datos \emph{raster} y topologías tridimensionales.

\subsection{Análisis, procesamiento y consultas espaciales}

%TODO footnote para PostGIS y Oracle Spatial

Una consulta espacial en una base de datos es aquella que utiliza funciones geométricas para responder preguntas acerca del espacio y de los objetos en el espacio. Extensiones para bases de datos relacionales tales como PostGIS y Oracle Spatial agregan un conjunto de funciones a las ya existentes en el lenguaje SQL estándar que trabajan con objetos geométricos de forma similar a cómo las funciones estándar trabajan con los tipos de datos nativos de la base de datos.

Por ejemplo, existen funciones que indican la cantidad de tiempo que hay entre dos fechas o si una fecha dada corresponde al pasado o al futuro. De forma similar las bases de datos espaciales proveen funciones que permiten obtener la distancia entre dos figuras geométricas, o el área total de un polígono. Las funciones espaciales también permiten crear y modificar objetos geométricos.

%TODO footnote para OGC

Actualmente existen estándares que definen este conjunto de tipos de datos y funciones espaciales mínimas para una base de datos GIS. El \emph{Open Geospatial Consortium} (OGC) es un consorcio internacional de la industria cuyo objetivo es tratar de estandarizar la forma en que los datos geométricos y espaciales son accedidos y distribuidos. OGC cuenta con numerosas especificaciones que definen el acceso a los datos geoespaciales, servicios web para consulta y manipulación de datos, formatos de datos geoespaciales y consulta de datos geoespaciales.

\subsection{Productos comerciales y de código abierto}

%TODO footnote para SQLite, MySQL, SQL Server, PostGIS

Existen varias bases de datos espaciales disponibles en el mercado que implementan los estándares OGC, tanto de código abierto como soluciones comerciales. Entre las bases de datos de código abierto más populares podemos citar a SpatialLite/SQLite, PostgreSQL/PostGIS y MySQL. Los productos comerciales más populares incluyen a Oracle Spatial y SQL Server de Microsoft.

PostGIS es una extensión para PostgreSQL que agrega soporte para tipos de datos geométricos y funciones espaciales convirtiendo a PostgreSQL en una base de datos espacial. PostGIS es un proyecto de código abierto que se publica bajo la Licencia Pública General de GNU. Además de los tipos de datos básicos provee otros tipos de datos más complejos como multi-polígonos, multi-puntos, multi-líneas y curvas geométricas. PostGIS 2 además agrega soporte para datos \emph{raster} y datos tridimensionales en formato TIN. De forma similar SpatialLite es una extensión para SQLite que agrega soporte para tipos de datos y funciones geométricas.

Oracle Spatial es una extensión para la base de datos Oracle que proporciona tipos de datos y funciones para la gestión de datos espaciales. Los tipos estándar y funciones de Oracle (CHAR, DATE, INTEGER, etc) se extienden con equivalentes geográficos. Oracle Spatial provee tres tipos básicos de figuras geométricas: puntos, líneas y polígonos (incluyendo polígonos complejos con agujeros), a partir de los cuáles se pueden construir estructuras de datos más complejas. Además, Oracle Spatial puede almacenar y administrar datos de tipo \emph{raster}.
