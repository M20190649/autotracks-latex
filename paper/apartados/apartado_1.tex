\section{Introducción}

La congestión del tránsito es uno de los problemas más serios que enfrentan la mayoría de las zonas urbanas hoy en día, el constante aumento del parque de automóviles, el alto uso de vehículos privados y la falta de planificación de las ciudades son algunos de los factores que afectan negativamente la calidad de vida de los ciudadanos. Además, los países en vías de desarrollado carecen de sistemas de control de tránsito debido al alto costo de inversión requerido y en algunos casos tampoco existen servicios proveídos por empresas privadas para este fin. Esta situación dificulta la utilización eficiente de las vías de tránsito existentes, es por eso que se hace evidente la necesidad de recolectar y analizar información sobre el tránsito de forma barata y sencilla. Los sistemas de información de tráfico basados en \emph{Floating Car Data} (FCD) han demostrado ser una alternativa viable para obtener esta información de forma económica y eficiente \cite{schafer2002traffic,reinthaler2007evaluation}.

Los sistemas existentes de FCD se basan en la utilización de vehículos sonda equipados con sensores GPS. Para este propósito se han utilizado flotas de taxis \cite{schafer2002traffic,reinthaler2007evaluation} y sistemas instalados por empresas de seguro en los vehículos de sus clientes \cite{giovannini2011novel}. Sin embargo, en países menos desarrollados no existen programas o iniciativas públicas y/o privadas que promuevan la utilización de este tipo de sistemas.

Otra alternativa para recolectar FCD es la utilización de teléfonos móviles inteligentes, equipados con sensores GPS, ya que los automovilistas generalmente llevan consigo este tipo de teléfonos. Este tipo de tecnología ya se ha utilizado en aplicaciones para el rastreo de vehículos \cite{thiagarajan2010cooperative}, estimación del tiempo de llegada de buses \cite{zhou2012long} y estimación de tiempo de viaje \cite{thiagarajan2009vtrack}. Diversos estudios han demostrado la factibilidad de utilizar esta tecnología para estimar el estado del tráfico en tiempo real \cite{tao2012real,herrera2010evaluation}, sugiriendo que una penetración de entre un 2 y 3\% es suficiente para proporcionar mediciones precisas de la velocidad del flujo del tráfico \cite{herrera2010evaluation}.

Para estimar el estado del tráfico primeramente se debe determinar la trayectoria de los vehículos por las calles de la ciudad, este proceso es conocido por el nombre de \emph{Map Matching} (MM). Existe una gran variedad algoritmos de MM, desde los más sencillos, basados solamente en información geográfica \cite{white2000some}, hasta los más complejos, basados en modelos estadísticos y otras técnicas avanzadas \cite{quddus2006high,kim2001adaptive}. Debido a las limitaciones impuestas por las plataformas móviles, tales como restricciones de consumo de batería, conectividad limitada o intermitente, errores de precisión, entre otros, resulta necesaria la utilización de algoritmos de MM especializados en procesar muestras relativamente dispersas y poco precisas \cite{lou2009map}.

En este trabajo se describe la implementación de un sistema que permite aproximar las condiciones del tránsito utilizando información proveída por dispositivos móviles en tiempo real. La solución propuesta está basada en FCD obtenido mediante una aplicación móvil desarrollada para el efecto, denominada \emph{Autotracks}, que se instala en los dispositivos de los usuarios y envía periódicamente la información obtenida a un servidor central. Para minimizar el impacto en el consumo de batería, se utiliza un mecanismo de reconocimiento de actividad que permite rastrear al usuario sólo cuando se encuentra en un vehículo en movimiento. Las trayectorias recolectadas son procesadas en el servidor utilizando un algoritmo de MM. La información resultante es almacenada en una base de datos GIS y luego es agregada para obtener una aproximación del estado del tráfico.

El resto del trabajo está organizado de la siguiente forma:
\begin{itemize}
\item El \Cref{sec:recoleccion_datos} describe las técnicas de recolección de información de tráfico. 
\item En el \Cref{sec:map_matching} se describe el problema de MM y los distintos tipos de algoritmos existentes en la actualidad. 
\item En el \Cref{sec:medidas_trafico} se explican los diferentes parámetros que caracterizan el flujo de tráfico. 
\item En el \Cref{sec:arquitectura} se presenta la solución propuesta, su arquitectura, las técnicas seleccionadas para la recolección y el análisis de datos, y los algoritmos utilizados en cada paso del proceso. 
\item En el \Cref{sec:pruebas} se describen las pruebas realizadas y se realiza un análisis de los resultados para verificar el funcionamiento del sistema. 
\item Finalmente, en el \Cref{sec:conclusiones} se exponen las conclusiones finales y trabajos futuros.
\end{itemize}
