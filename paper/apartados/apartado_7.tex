\section{Conclusiones}
\label{sec:conclusiones}

En este trabajo se describió la implementación de un sistema inteligente de información de tráfico de bajo costo y adecuado a las condiciones de infraestructura de los países en vías de desarrollo. Los datos de FCD son obtenidos mediante dispositivos móviles, utilizando un esquema de reconocimiento de actividad para rastrear los dispositivos únicamente cuando éstos se encuentran dentro de un vehículo en movimiento. Para el procesamiento de la información se utilizó el algoritmo ST-Matching, diseñado para trabajar con las muestras de baja frecuencia recolectadas por los dispositivos.

El volumen de datos recolectados durante las pruebas fue suficiente para determinar las horas pico y las velocidades promedio. La información recolectada muestra que los días entre semana son los que registran un mayor volumen de tráfico. Además, se identificó que el mayor volumen se da de 6 a 9 de la mañana, al mediodía y entre las 18 y 21 horas. Durante las horas pico en las que se registra un mayor volumen de tráfico también se aprecia una disminución considerable en la velocidad promedio de los vehículos.

En base a las pruebas de campo realizadas se puede concluir que los dispositivos móviles pueden ser utilizados para recolectar información del estado del tránsito en países que no cuentan con infraestructura instalada para el efecto. Además, la ubicación de los vehículos puede ser obtenida mediante cualquier sensor del dispositivo móvil, como ser los sensores GPS, Wifi y las redes de telefonía. Los datos obtenidos mediante Wifi o redes de telefonía son suficientemente aceptables.

Todos los componentes utilizados en la implementación de este sistema son libres y gratuitos. Además el código fuente, tanto de la aplicación web como de la aplicación móvil, está disponible con la Licencia MIT, que permite su libre utilización, distribución y modificación. Toda la información generada por el sistema está almacenada y disponible de manera gratuita para su utilización en trabajos futuros.