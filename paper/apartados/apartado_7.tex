\section{Conclusiones}
\label{sec:conclusiones}

En este trabajo se describió la implementación efectiva de un sistema inteligente de información de tráfico, de bajo costo y adecuado a las condiciones de infraestructura de los países en vías de desarrollo. Los datos de FCD son obtenidos mediante dispositivos móviles, utilizando un esquema de reconocimiento de actividad para rastrear los dispositivos únicamente cuando éstos se encuentran dentro de un vehículo en movimiento.

Los datos recolectados durante las pruebas permitieron generar una aproximación del tráfico en la que es posible determinar horas pico y sus velocidades promedio y que además está en concordancia con la realidad percibida diariamente. Se identifica que el mayor volumen de tráfico se da de 6 a 9 de la mañana, al mediodía y entre las 18 y 21 horas, y durante estos periodos se observa una disminución considerable en la velocidad promedio de los vehículos.

También para algunos segmentos de calles en donde se contó con usuarios suficientes y constantes día a día, fue posible proveer en tiempo real información acerca del estado de transito en los mismos.

En base a las pruebas de campo realizadas y los datos recolectados de usuario finales se observa la viabilidad de aplicar dispositivos móviles para aproximar el estado del tránsito cuando no se cuenta con infraestructura instalada para el efecto.

Como trabajos futuros se....