\section{Recolección de datos de tráfico}
\label{sec:recoleccion_datos}

Actualmente existen una variedad de tecnologías para la recolección automática de datos del tráfico. Se pueden dividir estas tecnologías en dos categorías. La primera es la \emph{tecnología in-situ}, que toma los datos del tráfico a través de detectores ubicados a lo largo del camino y que vuelve a dividirse en dos categorías: la \emph{intrusiva} y la \emph{no intrusiva}. La segunda, denominada \emph{Floating Car Data} (FCD), obtiene datos del tráfico mediante el uso de vehículos sonda equipados con dispositivos de medición \cite{mimbela2003summary}.

Las \emph{tecnologías de detección in-situ} se basan en la recolección de datos mediante dispositivos específicos para este fin, que son dispuestos físicamente en los lugares sujetos de medición. Se dividen en dos categorías: \begin{enumerate*}[a)] \item las tecnologías \emph{intrusivas}, que están montadas en o por debajo de la superficie de las rutas y cuya instalación ocasiona la interrupción potencial del tráfico (ej: tubos neumáticos y magnetómetros) y \item las tecnologías \emph{no intrusivas}, que son montadas encima o sobre la superficie de las rutas y cuya instalación no genera interrupción del tráfico, o lo hace en pequeña medida (ej: cámaras fotográficas o de video).\end{enumerate*}

Además de la utilización de tecnologías in-situ, muchas aplicaciones de gestión de tráfico utilizan dispositivos montados en vehículos en circulación, conocidos como sistemas de ubicación automática de vehículo (\emph{Automatic Vehicle Location} - AVL). Los dispositivos AVL pueden proveer dos tipos de información: \begin{enumerate*}[a)]
\item información de posición, cuando un vehículo equipado con ellos pasa cierto punto de la red donde existe un sensor, o \item información continua, a medida que el vehículo transita a través de la red.
\end{enumerate*}
Los vehículos sonda pueden ser equipados con dispositivos AVL dedicados o se pueden utilizar dispositivos móviles, tales como teléfonos celulares, que actúan como dispositivos AVL cuando su usuario viaja en vehículo.  

Un sistema basado en FCD consiste en recolectar y procesar datos de tráfico obtenidos en tiempo real, a través de dispositivos AVL montados en vehícuos que circulan en una red de caminos sujeta a observación. Todos los vehículos equipados con estos dispositivos actúan como sensores dentro de la red de caminos. Datos como la ubicación del vehículo, la velocidad y dirección del viaje son recabados para su procesamiento. Luego de la recolección y extracción de los datos, informaciones útiles tales como el estado del tráfico y rutas alternativas, pueden ser derivadas y posteriormente distribuidas.

