\selectlanguage{spanish}

\begin{abstract}
El creciente aumento del parque automotor en los países en vías de desarrollo representa un grave problema debido a que la ciudades no cuentan con infraestructura vial bien planificada y tampoco con tecnología instalada para monitorear el estado del tráfico, lo cual dificulta el tránsito diario y deteriora la calidad de vida de los ciudadanos. Debido a esto se hace evidente la necesidad de recolectar y analizar información sobre el tránsito de forma barata y sencilla, y que requiera una mínima inversión. Aprovechando que la utilización de teléfonos móviles inteligentes, equipados con sensores GPS, está creciendo a un ritmo considerable surge la posibilidad de utilizar esta tecnología para obtener esta información de forma económica y eficiente. 

En este trabajo se presentan los detalles de implementación del sistema denominado Autotracks, que permite recolectar y procesar información del estado del tránsito vehicular a través de dispositivos móviles para aproximar el estado del tráfico en tiempo real.  Se utilizó un esquema de \emph{Floating Car Data} en combinación con técnicas de detección de actividad para monitorear la ubicación de los vehículos en movimiento. La trayectoria real de los vehículos por las calles de la ciudad es determinada mediante un proceso de \emph{Map Matching}. Todas las trayectorias son agregadas para aproximar el estado del tráfico en un período de tiempo dado. La información del estado del tráfico puede ser consultada en tiempo real a través de la aplicación móvil o la aplicación web desarrolladas para el efecto. Toda la información generada es almacenada en una base de datos GIS para su posterior utilización. 
\end{abstract}

\selectlanguage{english}

\begin{abstract}
The increasing amount of vehicles in developing countries represents a serious problem because the cities do not have well planned road infrastructure and do not have technology installed to monitor traffic conditions, which hinders the daily traffic and impairs the life quality of citizens. Because of this the need to collect and analyze information on traffic cheaply, easily and with a minimum investment becomes evident. Taking advantage that the use of smart phones equipped with GPS sensors, is growing at a substantial rate, the use of this technology emerges as a posibility to obtain this information in an economical and efficient way.

This paper presents the implementation details of the system called Autotracks, which allows collecting and processing of vehicular traffic information through mobile devices in order to approximate the traffic conditions in real time. A Floating Car Data approach was used in combination with activity recognition techniques to monitor the location of the moving vehicles. The actual path of vehicles through the city streets  is determined using a Map Matching process. All the trajectories are aggregated in order to approximate the traffic conditions on a given time period. The traffic status information can be viewed in real time through the mobile application or the web application developed for this purpose. All the generated information is stored in a GIS database for later use.
\end{abstract}

\selectlanguage{spanish}