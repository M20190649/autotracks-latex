\begin{abstract}
El creciente aumento del parque automotor en los países en vías de desarrollo representa un problema de difícil solución debido principalmente a que sus ciudades no cuentan con infraestructura vial bien planificada y tampoco con tecnología instalada para monitorear el estado del tráfico, lo cual dificulta el tránsito diario y deteriora la calidad de vida de los ciudadanos. Debido a esto se hace evidente la necesidad de recolectar y analizar información sobre el tránsito de forma barata y sencilla pero a la vez efectiva. Aprovechando que la utilización de teléfonos móviles inteligentes, equipados con sensores GPS, está creciendo a un ritmo considerable surge la posibilidad de utilizar esta tecnología para obtener información acerca del tráfico vehicular de forma eficiente y que posteriormente sirva de apoyo para la optimización del mismo. 

En este trabajo se presentan detalles de implementación del sistema denominado Autotracks, que permite recolectar y procesar información del tránsito vehicular a través de dispositivos móviles para aproximar el estado del tráfico en tiempo real. Se utilizó un esquema de \emph{Floating Car Data} en combinación con técnicas de detección de actividad para monitorear la ubicación de los vehículos en movimiento tomándose muestras de su recorrido. La trayectoria real de los vehículos por las calles de la ciudad es luego determinada mediante un proceso de \emph{Map Matching}. Luego, para un periodo de tiempo dado, todas las trayectorias determinadas pueden ser agregadas entre sí para aproximar el estado del tráfico en ese momento. Así, la información del estado del tráfico puede ser consultada en tiempo real a través de la aplicación móvil o la aplicación web desarrolladas para el efecto. Además, toda la información generada es almacenada en una base de datos GIS, lo que deja la posibilidad de poder aprovechar posteriormente esta información para otras finalidades tales como la planificación de acciones de mejoramiento del tránsito vehicular.
\end{abstract}

\begin{otherlanguage}{english}
\begin{abstract}
The increasing amount of vehicles in developing countries represents a difficult problem mainly because their cities do not have well-planned road infrastructure nor installed technology to monitor traffic conditions, this hinders the daily traffic and impairs the quality of life of citizens. Because of this, the need to collect and analyze traffic information cheaply and easily, yet effective, becomes clear. Taking advantage that the use of smart phones, equipped with GPS sensors, is growing at a considerable rate, the use of this technology emerges as an alternative to collect information on vehicular traffic efficiently, this information can later be used to support traffic optimization.

This paper presents the implementation details of the system called Autotracks, which allows collecting and processing information from vehicular traffic through mobile devices in order to approximate the traffic conditions in real time. A Floating Car Data approach was used in combination with activity recognition techniques to monitor the location of the moving vehicles and collect samples of its route. The actual path of the vehicles through the city streets is determined using a Map Matching process. Then, for a given time period, all the trajectories are aggregated in order to approximate the traffic conditions at that particular time. Finally, the traffic status information can be viewed in real time through the mobile application or the web application developed for this purpose. In addition, all the generated information is stored in a GIS database, leaving the possibility to use this information for other purposes, such as planning actions to improve vehicular traffic.
\end{abstract}
\end{otherlanguage}