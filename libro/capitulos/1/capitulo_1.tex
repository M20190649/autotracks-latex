\chapter{Introducción}
\label{cap:1}

\section{Motivación}

La congestión del tránsito es uno de los problemas más serios que enfrentan la mayoría de las zonas urbanas hoy en día, el constante aumento del parque de automóviles, el alto uso de vehículos privados y la falta de planificación de las ciudades son algunos de los factores que hacen parte del problema comentado y que afectan negativamente la calidad de vida de los ciudadanos. Además, los países en vías de desarrollado carecen de sistemas de control de tránsito debido al alto costo de inversión requerido y en algunos casos tampoco existen servicios proveídos por empresas privadas para este fin. Esta situación dificulta la utilización eficiente de las vías de tránsito existentes, es por eso que se hace evidente y oportuno encontrar una posibilidad de recolectar y analizar información sobre el tránsito de forma barata y sencilla.

En este trabajo se propone la implementación de un sistema inteligente que permita aproximar las condiciones del tránsito utilizando información proveída por dispositivos móviles en tiempo real. Para ello se estudia el estado del arte de las técnicas de recolección, análisis y distribución de este tipo información, y se seleccionan los mecanismos más adecuados a las necesidades y limitaciones del medio. También se pretende que dicha información pueda quedar disponible de forma a aprovecharla posteriormente para desarrollar soluciones que ayuden a mitigar la problemática de la congestión vehicular.

\section{Justificación y Antecedentes}

Debido a que los países en vías de desarrollo cuentan con poca o ninguna infraestructura vial bien planificada y tampoco cuentan con tecnología instalada para monitorear el estado del tráfico, se hace necesario buscar una alternativa económica para obtener información precisa y confiable, que permita aproximar la situación actual del tránsito vehicular en tiempo real. Es así, que los sistemas de información de tráfico basados en \emph{Floating Car Data} (FCD) han demostrado ser una alternativa viable para obtener esta información de forma económica y eficiente \citep{schafer2002traffic,reinthaler2007evaluation}.

Los sistemas existentes de FCD se basan en la utilización de vehículos sonda equipados con sensores GPS. Para este propósito se han utilizado flotas de taxis \citep{schafer2002traffic,reinthaler2007evaluation} y sistemas instalados por empresas de seguro en los vehículos de sus clientes \citep{giovannini2011novel}. Sin embargo en países menos desarrollados no existen programas o iniciativas públicas y/o privadas que promuevan la utilización de este tipo de sistemas.

Por otra parte, la utilización de teléfonos móviles inteligentes, equipados con sensores GPS, está creciendo a un ritmo considerable y los automovilistas generalmente llevan consigo este tipo de teléfonos. Es así que surge la posibilidad de utilizar esta tecnología como una forma barata y sencilla de recolectar información para estimar el estado del tránsito. Este tipo de tecnología ya se ha utilizado en aplicaciones para el rastreo de vehículos \citep{thiagarajan2010cooperative}, estimación del tiempo de llegada de buses \citep{zhou2012long} y estimación de tiempo de viaje \citep{thiagarajan2009vtrack}. Siguiendo la idea anterior, diversos estudios han demostrado la factibilidad de utilizar esta tecnología para estimar el estado del tráfico en tiempo real \citep{tao2012real,herrera2010evaluation}, sugiriendo que una penetración de entre un 2 y 3\% podría ser suficiente para proporcionar mediciones precisas de la velocidad del flujo del tráfico \citep{herrera2010evaluation}.

Para estimar el estado del tráfico primeramente se debe determinar o reconstruir la trayectoria de los vehículos por las calles de la ciudad, este proceso es conocido por el nombre de \emph{Map Matching} (MM). Existe una gran variedad algoritmos de MM, desde lo más sencillos, basados solamente en información geográfica \citep{white2000some}, hasta los más complejos, basados en modelos estadísticos y otras técnicas avanzadas \citep{quddus2006high,kim2001adaptive}. Debido a las limitaciones impuestas por las plataformas móviles (uso de batería, conectividad limitada, entre otros) que se utilizan para obtener los datos, este trabajo se enfoca principalmente en la utilización de algoritmos especializados en procesar muestras relativamente dispersas y poco precisas \citep{lou2009map}.

Con el objetivo de distribuir la información de tráfico, muchos sistemas actualmente se basan en la formación de redes ad-hoc entre los dispositivos móviles \citep{zhong2008disseminating,leontiadis2011effectiveness}. Esto dificulta que los datos obtenidos sean procesados de forma adecuada y distribuidos a otros posibles usuarios que se encuentren fuera de estas redes ad-hoc. En este trabajo se propone una arquitectura centralizada de manera a que todo el procesamiento de la información y la distribución de la información sean manejados por un servidor dedicado.

Existen implementaciones comerciales y/o libres similares a la solución propuesta, sin embargo estos productos generalmente no se encuentran disponibles en los países en vías de desarrollo y además en la literatura no existen trabajos que presenten todos los detalles de diseño, arquitectura y resolución de cuestiones implementativas a la problemática estudiada. Los trabajos anteriores se enfocan generalmente en una sola parte de la problemática, ya sea en la recolección de datos, el análisis o la distribución de la información de tránsito, dejando de lado el estudio de la arquitectura completa de los sistemas.

De esta forma se pretende contribuir al estado del arte al realizar un estudio completo de las técnicas utilizadas en la implementación de sistemas de información de tránsito, abarcando cada una de las partes involucradas, desde la recolección hasta la distribución de la información, y para cada parte, estudiando y aplicando las técnicas apropiadas publicadas en el estado del arte.

\section{Objetivos}

El principal objetivo de este proyecto es construir un sistema que permita recolectar y procesar información del estado del tránsito vehicular a través de dispositivos móviles para aproximar el estado del tráfico en tiempo real.

Entre los objetivos específicos se puede citar:

\begin{itemize}

\item Analizar el estado del arte de las técnicas de recolección de datos de tránsito con un enfoque particular en el área de FCD.

\item Analizar el estado del arte en técnicas de MM utilizadas para procesar la información obtenida y reconstruir las trayectorias de los vehículos y sus propiedades.

\item Diseñar la arquitectura para un sistema que utilice las técnicas estudiadas de modo a recolectar, analizar, distribuir y utilizar la información del estado del tránsito vehicular.

\item Implementar una aplicación móvil que permita recolectar datos de ubicación y trayectoria de los usuarios, distribuir la información de tránsito y brindar soporte a los usuarios.

\item Implementar un sistema centralizado capaz de recibir los datos de ubicación y trayectoria de los usuarios para determinar el estado del tránsito vehicular.

\end{itemize}

\section{Organización}

El resto del presente trabajo se encuentra organizado de la siguiente forma. En el \Cref{cap:2} se presentan conceptos básicos sobre Sistemas de Información Geográfica y bases de datos espaciales.

El \Cref{cap:3} describe el estado del arte de las técnicas de recolección de datos, con un especial enfoque en las técnicas de FCD basadas en la utilización de dispositivos móviles. 

En el \Cref{cap:4} se realiza un extenso análisis del estado del arte de los algoritmos de MM utilizados para el procesamiento de la información recolectada.

En el \Cref{cap:5} se explican los parámetros, fórmulas y análisis realizados para derivar el estado del tráfico.

El \Cref{cap:6} describe detalladamente la arquitectura de la solución propuesta, las técnicas seleccionadas para la recolección y el análisis de datos, y los algoritmos utilizados en cada paso del proceso.

En el \Cref{cap:7} se presentan las pruebas realizadas y se realiza un análisis de los resultados para verificar el funcionamiento del sistema.

Finalmente en el \Cref{cap:8} se presentan las conclusiones y se proponen trabajos futuros que pueden ser llevados a cabo a partir de éste.
